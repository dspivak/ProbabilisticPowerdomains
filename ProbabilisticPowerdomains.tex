\documentclass[11pt, oneside, article]{memoir} 
\settrims{0pt}{0pt} % page and stock same size
\settypeblocksize{*}{32pc}{*} % {height}{width}{ratio}
\setlrmargins{*}{*}{1} % {spine}{edge}{ratio}
\setulmarginsandblock{1in}{1in}{*} % height of typeblock computed
\setheadfoot{\onelineskip}{2\onelineskip} % {headheight}{footskip}
\setheaderspaces{*}{1.5\onelineskip}{*} % {headdrop}{headsep}{ratio}
\checkandfixthelayout



\usepackage{mathtools}
\usepackage{amsthm}
\usepackage{amssymb}
\usepackage{stmaryrd}
\usepackage{bbm}
\usepackage{accents}
\usepackage{newpxtext}
\usepackage[utf8]{inputenc}
\usepackage[varg,bigdelims]{newpxmath}
\usepackage[usenames,dvipsnames]{xcolor}
\usepackage{tikz}
\usepackage{graphicx}
\usepackage{enumitem}
\usepackage{mathpartir}
\usepackage[bookmarks=true, colorlinks=true, linkcolor=blue!50!red, citecolor=orange,
pdfencoding=unicode]{hyperref}
\usepackage[capitalize]{cleveref}
  \newcommand{\creflastconjunction}{, and\nobreakspace}%Make cleveref use serial comma
\usepackage[backend=biber,style = alphabetic]{biblatex}
  \addbibresource{Library20171206.bib}

%%%% draft stuff
\usepackage[color=white]{todonotes}
\usepackage{showkeys}
\newcommand{\tob}[1]{\todo[color=blue!40,inline]{\textbf{T:} #1}}
\newcommand{\dis}[1]{\todo[color=red!40,inline]{\textbf{D:} #1}}



\usetikzlibrary{
	cd,
	math,
	decorations.markings,
	positioning,
	arrows.meta,
	shapes,
	calc,
	fit,
	quotes,
	intersections}
\hypersetup{final}
\setlist{nosep}

\tikzset{
  tick/.style={postaction={
    decorate,
    decoration={markings, mark=at position 0.5 with {\draw[-] (0,.4ex) -- (0,-.4ex);}}}
  },
  tickx/.style={
    postaction={ decorate,
      decoration={markings,
        mark=at position 0.5 with {
          \fill circle [radius=.28ex];
        }
      }
    }
  }
}



\theoremstyle{plain}
\newtheorem{theorem}{Theorem}[chapter] %change [] to chapter if we want to change global numbering
\newtheorem{proposition}[theorem]{Proposition}
\newtheorem{corollary}[theorem]{Corollary}
\newtheorem{lemma}[theorem]{Lemma}
\newtheorem{conjecture}[theorem]{Conjecture}

\theoremstyle{definition}
\newtheorem{definition}[theorem]{Definition}
\newtheorem{construction}[theorem]{Construction}
\newtheorem{notation}[theorem]{Notation}
\newtheorem{axiom}{Axiom}
\newtheorem{assumption}{Assumption}
\newtheorem*{axiom*}{Axiom}

\theoremstyle{remark}
\newtheorem{example}[theorem]{Example}
\newtheorem{remark}[theorem]{Remark}
\newtheorem{warning}[theorem]{Warning}
\newtheorem{question}[theorem]{Question}

% environment for soundness proofs
\newenvironment{soundproof}{\begin{proof}[Soundness proof]}{\end{proof}}

\setcounter{axiom}{0}

% Renewed commands

\renewcommand{\ss}{\subseteq}

% Macros %
\DeclarePairedDelimiter{\church}{\llbracket}{\rrbracket}
\DeclarePairedDelimiter{\Church}{\llbracket}{\rrbracket}
\DeclarePairedDelimiter{\subtype}{[}{]}
\DeclarePairedDelimiter{\classify}{\ulcorner}{\urcorner}

\DeclareMathOperator{\id}{id}
\DeclareMathOperator{\Hom}{Hom}
\DeclareMathOperator{\Mor}{Mor}
\DeclareMathOperator*{\colim}{colim}
\DeclareMathOperator{\im}{im}
\DeclareMathOperator{\Ob}{Ob}
\DeclareMathOperator{\interior}{int}

\newcommand{\const}[1]{\mathtt{#1}}
\newcommand{\Set}[1]{\mathrm{#1}}
\newcommand{\cat}[1]{\mathcal{#1}}
\newcommand{\Cat}[1]{\mathbf{#1}}
\newcommand{\fun}[1]{\mathit{#1}}
\newcommand{\Fun}[1]{\mathsf{#1}}

\newcommand{\smset}{\Cat{Set}}


\newcommand{\tickar}{\begin{tikzcd}[baseline=-0.5ex,cramped,sep=small,ampersand replacement=\&]{}\ar[r,tick]\&{}\end{tikzcd}}
\newcommand{\xtickar}[1]{\stackrel{#1}{\tickar}}
\newcommand{\cocolon}{:\!}
\newcommand{\iso}{\cong}
\newcommand{\To}[1]{\xrightarrow{#1}}
\newcommand{\Too}[1]{\xrightarrow{\;\;#1\;\;}}
\newcommand{\from}{\leftarrow}
\newcommand{\From}[1]{\xleftarrow{#1}}
\newcommand{\Fromm}[1]{\xleftarrow{\;\;#1\;\;}}
\newcommand{\surj}{\twoheadrightarrow}
\newcommand{\inj}{\rightarrowtail}
\newcommand{\wavyto}{\rightsquigarrow}

\newcommand{\tn}[1]{\textnormal{#1}}
\newcommand{\ol}[1]{\overline{#1}}
\newcommand{\ul}[1]{\underline{#1}}
\newcommand{\wt}[1]{\widetilde{#1}}
\newcommand{\wh}[1]{\widehat{#1}}
\newcommand{\ubar}[1]{\underaccent{\bar}{#1}}
\newcommand{\subsing}[1]{\mathrm{subsing}(#1)}	% subsingleton predicate
\newcommand{\conn}[1]{\mathrm{conn}(#1)}	% connectedness predicate

\newcommand{\internal}[1]{\raisebox{-.03ex}{$\mathbbmtt{#1}$}}
\newcommand{\hs}{\hspace{1.1pt}}


\newcommand{\EE}{\mathbb{E}} % expectation value
\newcommand{\II}{\mathbb{II}} % interval domain
\newcommand{\IR}{\mathbb{IR}} % interval domain
\newcommand{\NN}{\mathbb{N}}
\newcommand{\PP}{\mathbb{P}}
\newcommand{\QQ}{\mathbb{Q}}
\newcommand{\RR}{\mathbb{R}}
\newcommand{\VV}{\mathbb{V}}
\newcommand{\ZZ}{\mathbb{Z}}
\newcommand{\LR}{\ul{\mathbb{R}}}

\newcommand{\tNN}{\internal{N}\hs}
\newcommand{\tQQ}{\internal{Q}\hs}
\newcommand{\tQQp}{\tQQ_{+}}
\newcommand{\tZZ}{\internal{Z}\hs}
\newcommand{\tQQub}{\QQ^\infty}
\newcommand{\tRR}{\internal{R}\hs}
\newcommand{\tIR}{\internal{I\hs R}\hs}
\newcommand{\tII}{\bar{\ubar{\tRR}}\hs}
\newcommand{\tLR}{\ubar{\tRR}\hs}
\newcommand{\tUR}{\bar{\tRR}\hs}
\newcommand{\tRRub}{\tRR^\infty}
\newcommand{\tIRub}{\internal{I\hs R}^\infty}
\newcommand{\tLRub}{\ubar{\tRR}^{\infty}}
\newcommand{\tURub}{\bar{\tRR}^{\infty}}
\newcommand{\tIIub}{\bar{\ubar{\tRR}}^{\infty}}

\newcommand{\tRRat}[1]{\tRR_{\SeeInline{#1}}}


\newcommand{\tConst}{\mathtt{C}}
\newcommand{\ShFun}[1]{\mathrm{Fn}(#1)}

\newcommand{\Ind}[1]{\Fun{Ind}\tn{-}#1}
\newcommand{\Psh}{\Fun{Psh}}
\newcommand{\Shv}{\Fun{Shv}}
\newcommand{\Cont}{\Fun{Cont}}
\newcommand{\Idl}{\Set{Idl}}
\newcommand{\yoneda}{\Fun{y}}

\newcommand{\Prop}{\const{Prop}}
\newcommand{\Time}{\const{Time}}
\newcommand{\Pt}{\const{Pt}}
\newcommand{\cc}{\const{cc}}
\newcommand{\unit}{\const{1}}
\newcommand{\Poset}{\Cat{Poset}}
\renewcommand{\Top}{\Cat{Top}}
\newcommand{\Op}{\Set{Op}}
\renewcommand{\C}{\Cat{C}}
\newcommand{\Sub}{\Set{Sub}}
\newcommand{\pt}{\Fun{pt}}
\newcommand{\lsc}{\Fun{LsC}}

\newcommand{\op}{^\tn{op}}
\newcommand{\el}[1]{\tn{el}#1}
\newcommand{\asSh}{\Fun{Sh}} % sheafified object
\newcommand{\toSh}{\Fun{sh}} % sheafification map from object to sheafified object

\newcommand{\inv}{^{-1}}

\newcommand{\apart}{\,\#\,}
\newcommand{\restrict}[2]{#1\big|\hspace{0in}_{#2}}
\newcommand{\restrictsm}[2]{#1|\hspace{0in}_{#2}}
\newcommand{\BaseTopos}{\mathcal{B}}
\newcommand{\BaseSpace}{B}
\newcommand{\CB}{C\BaseSpace}
\newcommand{\Const}{\Fun{Const}}
\newcommand{\Sky}{\Fun{Sky}}

\newcommand{\Pointwise}{\pi}
\newcommand{\AtSymbol}{{@}}
\newcommand{\SeeSymbol}{{\down}}  % Old: \xi
\newcommand{\InSymbol}{{\upclose}}% Old: \iota
\newcommand{\At}[2][]{\AtSymbol^{#1}_{#2}}
\newcommand{\See}[2][]{\SeeSymbol^{#1}_{#2}}
\newcommand{\In}[2][]{\InSymbol^{#1}_{#2}}
\newcommand{\AtInline}[1]{@{#1}}
\newcommand{\SeeInline}[1]{\SeeSymbol{#1}}
\newcommand{\InInline}[1]{\InSymbol{#1}}


\newcommand{\sqss}{\sqsubseteq}
\newcommand{\specupclose}{{\uparrow}}
\newcommand{\specdownclose}{{\downarrow}}
\newcommand{\upclose}{{\rotatebox[origin=c]{90}{$\twoheadrightarrow$}}}
\newcommand{\downclose}{{\rotatebox[origin=c]{90}{$\twoheadleftarrow$}}}
\newcommand{\down}{\mathord{\downarrow}}
\newcommand{\up}{\mathord{\uparrow}}

\newcommand{\imp}{\Rightarrow}
\renewcommand{\iff}{\Leftrightarrow}
\newcommand{\true}{\const{true}}
\newcommand{\Bool}{\Set{Bool}}
\newcommand{\ev}{\fun{ev}}

\newcommand{\Span}{\Cat{Span}}
\newcommand{\set}{\text{--}\smset}

\newcommand{\adj}[5][30pt]{%[size] Cat L, Left, Right, Cat R.
\begin{tikzcd}[ampersand replacement=\&, column sep=#1]
  #2\ar[r, shift left=5pt, "{#3}"]\ar[r, phantom, "\Rightarrow" yshift=-.4pt]\&
  #5\ar[l, shift left=5pt, "{#4}"]
\end{tikzcd}
}

\newcommand{\adjr}[5][30pt]{%[size] Cat R, Right, Left, Cat L.
\begin{tikzcd}[ampersand replacement=\&, column sep=#1]
  #2\ar[r, shift left=5pt, "{#3}"]\ar[r, phantom, "\Leftarrow" yshift=-.6pt]\&
  #5\ar[l, shift left=5pt, "{#4}"]
\end{tikzcd}
}


\newcommand{\erase}[1]{}

\linespread{1.2}
\setsecnumdepth{subsection}
\settocdepth{section}
\setlength{\parindent}{15pt}

%%%%%%%%%%%%% Document %%%%%%%%%%%%%
\begin{document}

\title{Stochastic Processes as Internal Probability Spaces\\ Or: A Synthetic Theory of Stochastic Processes}

\author{Tobias Fritz and David I.\ Spivak\thanks{}}

\maketitle

\todo[inline]{Tentative suggestions for title. Internal ``probability spaces'' may be technically misleading, but has the advantage of being more recognizable}

\tableofcontents*


%%%%%%%%% Chapter %%%%%%%%%

\chapter{David question}
Let $(X,F)$ be an internal topological space in the temporal topos (so every ``open set" $f:F$ is of the form $f\colon X\to \Prop$), and let $\tLR$ denote the lower reals.

Externally, one might define a probability measure on $F$ to be a map of sheaves $\mu\colon F \to \tLR$ such that 
\begin{enumerate}
	\item for all lengths $\ell$, the function $\mu(\ell)\colon F(\ell)\to\tLR(\ell)$ is a probability valuation; and
	\item for subintervals $\ell'\ss\ell$, the restriction map $F(\ell)\to F(\ell')$ is measure preserving.
\end{enumerate}
I think this is generally stronger than the internal condition, that $\mu$ be an internal probability valuation. If every $f:F$ is clopen, i.e. has a complementary open set, then I think the two notions are equivalent. 

I think we can add an additional axiom to the internal characterization to obtain the external one.

Assume that for any $p:\Prop$ and open set $f:F$, the set $(p \imp f)$ is again open. I may not need quite so strong an assumption. Then I conjecture that an internal valuation $\mu$ externally satisfies 1 and 2 iff it satisfies the proposition
\[\forall(t:\Time)(a,b:\tRR)(f:F).\In{[a,b]}\mu(f)=\mu(\In{[a,b]}f)\]
or written out without the jargon:
\[\forall(t:\Time)(a,b:\tRR)(q:\tQQ)(f:F).[a\leq t\leq b)\imp(q<\mu(f))]\iff q<\mu[(a\leq t\leq b)\imp f]
\]

\chapter{Introduction}

\chapter{Logical axioms and their soundness}

\section{The upper space}

\tob{Let's see at the end whether our proofs go through more generally still, i.e.~for spaces that are not necessarily spaces of compacts}

For a topological space $\BaseSpace$, we denote by $\CB$ its set of nonempty compacts, where if $K\ss\BaseSpace$ is compact, we denote the associated point by $K_{C\BaseSpace}\in C\BaseSpace$. We equip $C\BaseSpace$ with what is known as the \emph{upper space topology} of $B$~\cite{Edalat:1995a}. Namely, to each open $U\subseteq \BaseSpace$ there is an associated basic open in $C\BaseSpace$, denoted $U_{C\BaseSpace}$, consisting of those points $K_{C\BaseSpace}$ with $K\subseteq U$:
\[
	U_{C\BaseSpace}\coloneqq\{K_{C\BaseSpace}\mid K\ss U\}.
\]

For any $K\in CB$, we write $\At{K} := ((- \imp K^c) \imp K^c)$ for the associated quasi-closed modality, where $K^c$ denotes the interior of the complement of the point $\{K\}\in\CB$. 

\tob{The following is probably irrelevant:}

If $\BaseSpace$ is Hausdorff, then the closure of a point $K\in C\BaseSpace$ is
\[\down K=\{K'\in C\BaseSpace\mid K\ss K'\}.\] 
It is closed since any $K'\not\supseteq K$ can be separated from a chosen $b\in K\setminus K'$ by an open, and it is clearly the smallest closed set containing $K$. It follows that the specialization preorder is reverse containment\footnote{The Sierpinski space shows that this is not true in general.}, and it is directed complete~\cite[Proposition~3.1]{Edalat:1995a}.

\todo[inline]{I'm not sure if the topology necessarily coincides with the Scott topology. Prop 3.1(iii) of Edalat suggests that it doesn't. But perhaps in the locally compact case?}

\begin{lemma}
For $K\in C\BaseSpace$, the complement $(\down K)^c$ of $\down K$ is a prime open.
\label{lem.prime}
\end{lemma}

\begin{proof}
Suppose that we have opens $P,Q\subseteq C\BaseSpace$ such that $(P\cap Q)\cap \down K = \emptyset$. This implies $K\not\in P$ or $K\not\in Q$; suppose the former. Since an open is up-closed in the domain order, we have $P\cap \down K = \emptyset$, as was to be shown.
\end{proof}

\begin{lemma}
	\label{lem.locally_connected}
	$\CB$ is locally connected.
\end{lemma}

\begin{proof}
	For every $K$ contained in an open $U$, we need to find a connected open neighborhood of $K$ contained in $U$. Since $U$ is open, there is $\wh{U} \subseteq \BaseSpace$ such that $K\in\wh{U}$ and every $K'\subseteq \wh{U}$ is in $U$. Now the set of all compacts contained in $\wh{U}$ is a basic neighborhood of $K$ contained in $U$, and it is connected since its specialization preorder is connected.
\end{proof}

\begin{corollary}
	\label{cor.open_components}
	The connected components of every open in $\CB$ are themselves open.
\end{corollary}

\begin{proof}
	This follows from local connectedness.
\end{proof}

\section{Semantics of the $\At{K}$-modality}

\begin{proposition}
If $K$ is a point then the topos of sheaves for the associated modality $\At{K}$ is a copy of $\smset$.
\end{proposition}
\begin{proof}
A subsheaf $X\inj Y$ is $\At{K}$-dense if for every open $U$ in $C\BaseSpace$, either $K\not\in U$ or for every section $y\in Y(U)$, there exists $V\ss U$ with $K\in V$ and $\restrict{y}{V}\in X$. Recall that $F$ is an $\At{K}$-sheaf iff $F\to 1$ is right-orthogonal to every $\At{K}$-dense map.

Suppose $F$ is an $\At{K}$-sheaf; we will show that it is a skyscraper sheaf at $K$, i.e.\ there is a set $S$ such that for any open $U$, if $K\in U$ then $F(U)=S$ and if $K\not\in U$ then $F(U)=\{*\}$. Indeed let $S\coloneqq F(C\BaseSpace)$. For any $U$, if $K\in U$ then $\yoneda U\to\yoneda X$ is $\At{K}$-dense, and right-orthogonality means that the restriction map $F(X)\to F(U)$ is an iso. If $K\not\in U$ then $\emptyset\to\yoneda U$ is dense, and right-orthogonality means that $F(U)\to 1$ is an iso. Thus $F$ is the skyscraper on $S$.

The category of skyscraper sheaves at $K$ is equivalent to $\smset$.
\end{proof}

\section{Axioms and soundness proofs}

We work in a Grothendieck topos $\cat{E}$ (i.e.\ the category of sheaves on a site). In particular, it is equipped with a natural numbers object $\tNN$.

We only work with $\At{}$ modalities explicitly, since In and See are easy to define directly in terms of the logic, so that introducing notation for them would not decrease formula complexity by much.

In other words, we only equip standard first-order intuitionistic type theory with an additional system of modalities $\{ \At{K} \: : \: K\in \CB\}$ indexed by some type $\CB$. 

Our soundness proofs refer to the intended semantics.

We now axiomatize the assumption that our topos has enough points:

\begin{axiom}\label{ax.enough_points}
For every $P:\Prop$,
\[
	(\forall K\ldotp \At{K} P) \imp P.
\]
\end{axiom}

Since the right-hand side clearly implies the left, we conclude that this implication is actually an equality of propositions.


\begin{soundproof}
	We need to show that if $U \models \forall K\ldotp \At{K} P \imp \At{K} Q$, then also $U \models (P \imp Q)$.

	Proving the contrapositive, we assume $U \not\models (P \imp Q)$, which means that there is a point $K \in (U\cap P)\setminus Q$. Then we have $U \not\models (\At{K} P \imp \At{K} Q)$, which shows in particular that $U\not\models \forall K\ldotp \At{K} P \imp \At{K} Q$.
\end{soundproof}

\dis{Does the following work?}
\begin{soundproof}
	We need to show that if $U \models \forall K\ldotp \At{K} P$, then also $U \models P$.

	Proving the contrapositive, we assume $U \not\models P$, which means $U\not\ss P$; thus there is a point $K \in U\setminus P$. Then $P\cap U\ss(\down K)^c$, and since $U\not\ss (\down K)^c$, so since $\At{K}P=((P \imp K^c) \imp K^c)$, we have $U \not\models \At{K} P$ as desired.
\end{soundproof}


\begin{axiom}\label{ax.atBoolean}
	For every $K : \CB$ and $P : \Prop$,
\[
	\At{K} P \lor ( P \imp \At{K} \bot).
\]
\end{axiom}

Hence the topos of $\At{K}$-sheaves is Boolean.

\begin{soundproof}
	\tob{tbw}
\end{soundproof}

\begin{axiom}
	\label{ax.N_vs_at}
	For all $K:\CB$ and $P : \tNN \to \Prop$,
	\[
		\forall n : \tNN \ldotp P(n) \lor \At{K} \bot \quad \Longrightarrow \quad \At{K} \forall n : \tNN \ldotp P(n).
	\]
\end{axiom}

\begin{soundproof}
	Let $U\ss\CB$ be open. We need to show that if $U \models \forall n\ldotp P(n) \lor \At{K} \bot$, then also $U \models \At{K} \forall n : \tNN \ldotp P(n)$. The assumption says equivalently that for all $V\subseteq U$ and $n\in \tNN(V)$, we have $V \subseteq P(n) \cup (\down K)^c$. The claim is that if $K\in U$, then there is $W$ with $K\in W \subseteq U$ such that for all $W'\subseteq W$ and $n\in\tNN(W')$, we have $W' \ss P(n)$.

	So suppose $K\in U$, and consider the set
	\[
		W := \up (U \cap (\down K)).
	\]
		We show that $W$ is open, by proving that every $K'\in W$ has a neighborhood contained in $W$. By definition of $W$, there is $\wt{K} \supseteq K'$ with $\wt{K} \in U$ and $K \subseteq \wt{K}$. By openness of $U$, we have a $B$-open $\wt{U} \subseteq B$ such that $\wt{K} \subseteq \wt{U}$, and also every other $\wh{K} \subseteq \wt{U}$ is in $U$. Considering $\wt{U}$ as a basic open of $\CB$, we have $K'\in\wt{U}$, and for any other $\wh{K}\in\wt{U}$, we have $\wh{K}\in W$ because $\wh{K}\ss(\wh{K}\cup\wt{K})\supseteq K$ and $(\wh{K}\cup\wt{K})\in\wt{U}\ss U$. Hence $\wt{U}$ is the desired neighborhood of $K'$ which lies in $W$.

%	We show that $W$ is open, by proving that every $K'\in W$ has a neighborhood contained in $W$. By definition of $W$, there is $\wt{K} \supseteq K'$ with $\wt{K} \in U$ and $K \subseteq \wt{K}$. By openness of $U$, we have an open $\wt{U} \subseteq B$ such that $\wt{K} \subseteq \wt{U}$, and also every other $\wh{K} \subseteq \wt{U}$ is in $U$. But then also $\wh{K} \in W$, since we have $ \wh{K} \subseteq (\wh{K} \cup \wt{K}) \supseteq K$, and $\wh{K} \cup \wt{K} \in U$. Hence $\wt{U}$ is the desired neighborhood of $K'$ which lies in $W$.

	It follows that $W$ is the smallest open which satisfies \todo{a figure may help}
	\begin{equation}
		\label{UKW}
		U \cap \downarrow K = W \cap \downarrow K.
	\end{equation}
	We now show that $W$ is connected. So suppose that $W = W_1 \amalg W_2$, say with $K\in W_1$. Then every putative $K' \in W_2$ is such that there is $\wt{K} \supseteq K$ with $\wt{K} \supseteq K'$. Since these containments are specialization preorder relations, the first can hold only if $\wt{K} \in W_1$, and the second one only if $\wt{K} \in W_2$, which is a contradiction. Hence our assumption that there is $K' \in W_2$ was wrong, and we conclude $W_2 = \emptyset$, so that $W$ is connected.

	We now prove the claim that for all $W' \subseteq W$ and $n \in \tNN(W')$, we have $W' = P(n)$. Suppose first that $W'$ is connected. Then the restriction map $\tNN(W) \to \tNN(W')$ is the identity of $\NN$, so that it is enough to consider the case $W' = W$. But then we can take $V := W$ in the assumption, getting $W \subseteq P(n) \cup (\downarrow K)^c$. The definition of $W$ now implies that $U \cap \downarrow K \subseteq P(n)$. And since $P(n)$ is up-closed, we conclude $W \subseteq P(n)$, as was to be shown.

	Finally, for arbitrary $W'$, we have a decomposition $W' = \coprod_{i\in I} W_i$ with open components $W_i$ by \Cref{cor.open_components}. Every $n \in \tNN(W')$ is therefore given by a tuple $n = (n_i)_{i\in I}$ for $n_i \in \tNN(W_i) = \NN$. We prove the claim $W' \subseteq P(n)$ by showing $W_i \subseteq P(n)$ for every $i$; but then it is enough to prove $W_i \subseteq P(n_i)$, which we already know per the connected case from the previous paragraph.
\end{soundproof}

Finally, we have one axiom schema indexed by the types $X$.

\begin{axiom}
	\label{ax:flabby_vs_at}
	Let $K : \CB$ and suppose that $X$ is $\At{K}$-flabby. Then for all predicates $P : X \to \Prop$,
	\[
		\At{K}\exists x:X. P(x) \quad \Longrightarrow \quad \exists x:X.\At{K}P(x).
	\]
\end{axiom}

Since the implication from right is left is easy to prove for every modality, we therefore know that this is actually an equivalence.

\begin{soundproof}
	We need to show that if $U \models \At{K} \exists x:X.P(x)$, then $U \models \At{K}\bot \:\lor\: \exists x:X.\At{K}P(x)$. For $K\not\in U$, then the latter is trivially correct since $\At{K}$-flabby assumes inhabited. Hence we suppose $K \in U$. Then the assumption means that there is $V\ni K$ with $V\subseteq U$ and $V\models \exists x:X.\At{K} P(x)$. Restricting further, we have $W\ni K$ with $W\subseteq V\subseteq U$ and $x\in X(W)$ with $W \models \At{K} P(x)$. Restricting $W$ again if necessary, we get $W \models P(x)$.

	We now use the characterization of $\At{K}$-flabbiness of \cref{prop.at_flabby_semantics} to get $x'\in X(U)$ which represents the same germ at $K$ as $x$. Then we indeed have $U\models \At{K} P(x')$ since $W\models P(x)$.
\end{soundproof}

\tob{Todo: explain that the axiom does not hold without the flabbiness assumption}

\chapter{Consequences of the axiomatics}

\section{General properties of the modalities}

\begin{lemma}
	\label{lem.at_double_neg}
	For any $K$ and $P : \Prop$,
	\[
		\At{K} P \: = \: \left( (P \imp \At{K} \bot) \imp \At{K} \bot \right) .
	\]
\end{lemma}

\begin{proof}
	The left-to-right implication is straightforward and holds for every modality. For the right-to-left implication, we can assume $P \imp \At{K} \bot$ by the decidability \cref{ax.atBoolean}. But then we get $\At{K} \bot$ by assumption, and of course it implies $\At{K}P$.
\end{proof}

\begin{lemma}
For every $P,Q : \Prop$,
\[
	\left(\forall K\ldotp \At{K} P \imp \At{K} Q\right) \: \Longleftrightarrow \: \left(P \imp Q\right).
\]
\end{lemma}

\begin{proof}
	The implication from right to left is trivial. From left to right, by \cref{ax.enough_points}, it is enough to prove that if $P$, then $\At{K} Q$ for every $K$. But this follows from the assumption.
\end{proof}

The following observation is closely related to Axiom 9 of TTT. It can be interpreted as stating that if $P$ holds at $K$, then it also holds in a neighborhood of $K$.

\begin{proposition}
	\label{prop.at_to_nghbhd}
	For every predicate $P : X \to \Prop$,
	\[
		\At{K} P \: \Longleftrightarrow \: \exists U : X \to \Prop \ldotp \At{K} U \wedge (U \imp P).
	\]
\end{proposition}

\begin{proof}
	From left to right, take $U := P$. From right to left, combine the given assumptions.	
\end{proof}

\begin{lemma}
	\label{lem.at_vs_or}
	For every $P,Q \in \Prop$ and $K \in \CB$,
	\[
		\At{K}(P \lor Q) = \At{K}P \lor \At{K}Q.
	\]
\end{lemma}

\begin{proof}
	Use \Cref{lem.11_at_flabby} together with \Cref{ax:flabby_vs_at}.
\end{proof}

\begin{proposition}
	\label{prop.N_or_vs_forall}
	For every $P : \Prop$ and $Q : \tNN \to \Prop$,
	\[
		\forall n : \tNN\ldotp P \lor Q(n) \quad \Longleftrightarrow \quad P \lor \forall n : \tNN \ldotp Q(n),
	\]
	and likewise for $\tQQ$ in place of $\tNN$.
\end{proposition}

\begin{proof}
	Since $\tQQ$ and $\tNN$ are internally bijective, it is enough to work with $\tNN$.

	It is enough to prove the implication from left to right. Then by \Cref{ax.enough_points} and \Cref{lem.at_vs_or}, it is furthermore enough to derive
	\[
		\At{K} P \, \lor \, \At{K} \forall n : \tNN \ldotp Q(n).
	\]
	from the assumption. Now in case $\At{K} P$, we are done, so that we can assume $P \imp \At{K} \bot$ without loss of generality. Hence we actually have the assumption
	\[
		\forall n : \tNN \ldotp \At{K} \bot \lor Q(n),
	\]
	from which we then derive the desired $\At{K} \forall n : \tNN \ldotp Q(n)$ by \Cref{ax.N_vs_at}.
\end{proof}

\begin{proposition}
	Let $(X,\Op(X))$ be an internal space. Then for every $P : \Op(X)\to \Prop$ and $K : \CB$, we have
	\[
		\At{K} \exists (U : \Op(X))\ldotp P(U) \quad \Longleftrightarrow \quad \exists (U : \Op(X))\ldotp \At{K} P(U).
	\]
\end{proposition}

\begin{proof}
	\tob{we still need to show that flabbiness implies $\At{K}$-flabbiness}
	Combine \cref{ax:flabby_vs_at} with \cref{prop:opens_flabby}.	
\end{proof}

\section{Properties of the rationals}

The notion of flabbiness is defined in \cref{def:flabby} and that of $\At{K}$-flabbiness is defined in \cref{def.at_flabby}.

\begin{lemma}
	$\tNN$ is $\At{K}$-flabby for every $K : \CB$.
	\label{lem:N_flabby}
\end{lemma}

\begin{proof}
	We start with the tautology \dis{Why is this a tautology?}
	\[
		\forall n : \tNN \ldotp \left( P(n) \imp \At{K} \bot \: \lor \: \exists n' : \tNN \ldotp \At{K} P(n') \right),
	\]
	which by \Cref{prop.N_or_vs_forall} is equivalent to
	\[
		\left(\exists n : \tNN \ldotp \At{K} P(n) \right) \: \lor \: \left( \forall n : \tNN \ldotp P(n) \imp \At{K} \bot \right).
	\]
	Now in case that there is $n$ with $\At{K} P(n)$, there is nothing more to prove. Hence we assume $P(n) \imp \At{K} \bot$ for all $n$, which is similar to what we did in the proof of \cref{lem.11_at_flabby}. But now together with $\At{K} \exists n : \tNN \ldotp P(n)$, we arrive at $\At{K} \bot$, which is enough.
\end{proof}

\tob{The previous proof, as well as the one of \Cref{lem.11_at_flabby}, actually prove the claim of \Cref{ax:flabby_vs_at} directly. Turns this into the defn of $\At{K}$-flabby?}

\begin{lemma}
	$\tQQ$ is flabby.
	\label{lem:Q_flabby}
\end{lemma}

\begin{proof}
	Use any (internal) bijection between $\tNN$ and $\tQQ$ together with \cref{lem:N_flabby}.
	\tob{are we assuming/using that $\At{K}$-flabbiness for every $K$ implies flabbiness?}
\end{proof}

\begin{lemma}
	\label{lem:rational_compare_at}
	For $q,q'\in\tQQ$ and $K\in\CB$, we have
	\[
		\At{K}(q < q') = \At{K}\bot \, \lor \, (q < q'),
	\]
	and
	\[
		\At{K}(q = q') = \At{K}\bot \, \lor \, (q = q').
	\]
\end{lemma}

\begin{proof}
	We only prove the first equation since the second one is analogous.

	Focusing on the nontrivial direction, we need to argue that $\At{K}(q < q')$ implies $\At{K}\bot$ or $q < q'$. Using the decidability of inequalities between rationals, we can assume that $q \geq q'$. But then the assumption $\At{K}(q < q')$ together with $\At{K}(q \geq q')$ implies $\At{K}\bot$.
\end{proof}

\section{Properties of the lower reals}

\begin{lemma}\label{lemma.plus_def}
Let $a_1,a_2:\tLR$ be lower reals. The following are equivalent for $q:\tQQ$.
\begin{enumerate}
	\item $\exists(q_1,q_2:\tQQ)\ldotp (q<q_1+q_2)\wedge(q_1<a_1)\wedge(q_2<a_2)$
	\item $\exists(q_1,q_2:\tQQ)\ldotp (q=q_1+q_2)\wedge(q_1<a_1)\wedge(q_2<a_2)$
	\item $\exists(q_1,q_2:\tQQ)\ldotp (q\leq q_1+q_2)\wedge(q_1<a_1)\wedge(q_2<a_2)$
\end{enumerate}
\end{lemma}
\begin{proof}
Clearly $(1)\imp(3)$ and $(2)\imp(3)$, and we have $(3)\imp(1)$ by roundededness. For $(3)\imp(2)$, take $q_1',q_2'$ satisfying (3); the result follows by setting $q_1\coloneqq q-q_2'$, so $q_1\leq q_1'<a_1$, and $q_2\coloneqq q_2'$.
\end{proof}
For lower reals $a$ and $b$, we denote by $a+b$ any of the equivalent conditions in \cref{lemma.plus_def}. We also write $a\leq b$ if $q < a$ implies $q < b$ for $q\in\tQQ$. We also use the more intuitive notation $q < a$ in place of $a(q)$ for $a\in\tLR$ and $q\in\tQQ$. This is consistent notation in the sense that the $q$ in $q < a$ can also be interpreted as a lower real, and then $q < a$ is indeed equivalent to $a(q)$.

For $a\in\tLR$ and $P : \Prop$, let us write $P \lor a$ for the lower real with
\[
	(P \lor a)(q) := P \lor a(q).
\]
It is easy to see that $P \lor a$ is again down-closed, rounded, and inhabited, so that $P \lor a \in \tLR$. 

\begin{lemma}
	\label{lem.or_add_LR}
	For $a,b\in\tLR$ and $P : \Prop$,
	\[
		P \lor (a + b) = (P \lor a) + (P \lor b) = (P \lor a) + b.
	\]
\end{lemma}

\begin{proof}
	Applied to $q \in \tQQ$, the left-hand formula is
	\[
		P \lor \exists q_1, q_2  : \tQQ \ldotp (q \leq q_1 + q_2) \land a(q_1) \land b(q_2),
	\]
	while the middle formula is
	\[
		\exists q_1, q_2  : \tQQ \ldotp (q \leq q_1 + q_2) \land (P \lor a(q_1)) \land (P \lor b(q_2)),
	\]
	and the right-hand formula is
	\[
		\exists q_1, q_2  : \tQQ \ldotp (q \leq q_1 + q_2) \land (P \lor a(q_1)) \land b(q_2),
	\]
	It is easy to see that the first two are equivalent; the third trivially implies the second. The first implies the third: in case that $P$ holds, we can simply choose $q_2$ such that $b(q_2)$ holds, which is possible by inhabitedness, and then take $q_1 := q - q_2$.
\end{proof}

Moreover, for $a,b : \tLR$, we define min and max as
\[
	\min(a,b)(q) := a(q) \land b(q),\qquad	\max(a,b)(q) := a(q) \lor b(q).
\]
It is easy to see that both expressions are indeed lower reals.

\begin{lemma}
	\label{lem.max_min_LR}
	For $a,b : \tLR$,
	\[
		\min(a,b) + \max(a,b) = a + b.
	\]
\end{lemma}

\begin{proof}
	Applied to some $q : \tQQ$, the left-hand side is 	
	\[
			\exists q_-, q_+  : \tQQ \ldotp (q \leq q_- + q_+) \land (a(q_-) \land b(q_-)) \land (a(q_+) \lor b(q_+)),
	\]
	while the right-hand side is
	\[
			\exists q_a, q_b  : \tQQ \ldotp (q \leq q_a + q_b) \land a(q_a) \land b(q_b).
	\]
	From right to left, we can take $q_- := \min(q_a,q_b)$ and $q_+ := \max(q_a,q_b)$. From left to right, assume $a(q_+)$ without loss of generality; then taking $q_a := q_+$ and $q_b := q_-$ works.
\end{proof}

\begin{lemma}
	\label{lem:eps_order}
	For every $a,b\in\tLR$,
	\[
		\big(\forall \epsilon : \tQQ\ldotp (\epsilon > 0) \imp (a \leq b + \epsilon)\big) \quad \Longrightarrow \quad \left( a \leq b \right).
	\]
\end{lemma}

\begin{proof}
	We need to prove that $q < a$ implies $q < b$. We can choose $q' \in \tQQ$ such that $q < q' < a$. Applying the assumption to $\epsilon\coloneqq q' - q > 0$, we have $a \le b + q' - q$. Hence $q' < a$ implies $q' < b + q' - q$, or equivalently $q < b$, as was to be shown.
\end{proof}

We now turn to the interaction of $\At{K}$-modalities with lower reals. For $r\in\tLR$, we put $(\At{K}r)(q) := \At{K}(r(q))$, and as usual we may write $q<\At{K}r$.

\begin{lemma}
	$\At{K}r$ is again a lower real.	
\end{lemma}

\begin{proof}
	For down-closure, we need to prove that if $q < q'$ and $\At{K}(r(q'))$, then also $\At{K}(r(q))$, which is routine.

	Roundedness is a bit more tricky. We need to show $\forall q\ldotp \At{K} r(q) \imp \exists q'\ldotp (q < q') \land \At{K} r(q')$. By \cref{ax:flabby_vs_at} and \cref{lem:Q_flabby}, roundedness of $r$ shows that $\exists q'\ldotp \At{K}((q < q') \land r(q'))$. The claim now follows from commutation of $\At{K}$ with conjunction and \cref{lem:rational_compare_at}.
\end{proof}

\begin{lemma}
	\label{lem:lower_reals_local}
	For every $a : \tLR$,
	\[
		a = \inf_K \At{K} a.
	\]
\end{lemma}

\begin{proof}
	For every $q : \tQQ$, we have $q < a$ if and only if $\At{K}(q < a)$, or equivalently $q < \At{K}a$, for every $K$.
\end{proof}

\begin{proposition}\label{prop:enough_points_LR}
For any $a,b:\tLR$,
\[
	a \leq b \quad \Longleftrightarrow \quad \forall K\ldotp \At{K} a \leq \At{K} b.
\]
\end{proposition}

\begin{proof}
	We focus on the nontrivial direction from right to left. So let $q:\tQQ$ satisfy $q < a$. Then also $q < \At{K} a$ for every $K$, and therefore $q < \At{K}b$, or equivalently $\At{K}(q < b)$. But since $K$ was arbitrary, we conclude $q < b$ from $\forall K\ldotp \At{K}(q < b)$.
\end{proof}

\begin{lemma}
	\label{lem:at_totally_ordered1}
	For all $a : \tLR$ and $q : \tQQ$,
	\[
		 (\At{K} a \leq \At{K}q) \: \lor \: \At{K}(q < a).
	\]
\end{lemma}

\begin{proof}
	By the $\At{K}$-decidability, we can assume $(q < a) \imp \At{K}\bot$. Then we need to prove that for every $q' : \tQQ$, the implication $\At{K}(q' < a) \imp \At{K}(q' < q)$ holds. Using the decidability of inequalities between rationals, we can assume $\At{K}(q' \geq q)$ in addition to $\At{K}(q' < a)$. These combine to $\At{K}(q < a)$, from which $\At{K}\bot$ follows by assumption. This is enough to get $\At{K}(q' < q)$.
\end{proof}


\begin{lemma}
	\label{lem.at_plus_commute}
	For all $a,b\in\tLR$,
	\[
		\At{K}(a + b) = \At{K} a + \At{K} b.
	\]
\end{lemma}

\begin{proof}
	For $q\in\tQQ$, we have $q < \At{K}(a + b)$ if and only if
	\[
		\At{K}\left(\exists q_1, q_2\in\tQQ\ldotp (q = q_1 + q_2) \land (q_1 < a) \land (q_2 < b)\right),
	\]
	which by \cref{ax:flabby_vs_at}, \cref{lem:Q_flabby} and \cref{lem:rational_compare_at} is equivalent to
	\begin{equation}
		\label{eq:q1q2_at1}
		\exists q_1, q_2\in\tQQ\ldotp \left((q = q_1 + q_2) \lor \At{K} \bot\right) \land \At{K}(q_1 < a) \land \At{K}(q_2 < b).
	\end{equation}
	Similarly, we have $q < \At{K} a + \At{K} b$ if and only if
	\begin{equation}
		\label{eq:q1q2_at2}
		\exists q_1, q_2 \in \tQQ \ldotp (q = q_1 + q_2) \land \At{K}(q_1 < a) \land \At{K}(q_2 < b).
	\end{equation}
	This is easily seen to be equivalent to \eqref{eq:q1q2_at1}; the only nonobvious case is if we have $q_1, q_2$ and $\At{K} \bot$. In this case, we can e.g.~use the decomposition $q = q + 0$ in~\eqref{eq:q1q2_at2}.
\end{proof}

\begin{lemma}
	\label{lem:subtract_q}
	For all $q,q':\tQQ$ and $a,a' : \tLR$,
	\begin{enumerate}
	\item Adding rationals is \emph{cancelable}:
		\begin{gather*}
			q + q' < a + q' \quad \Longrightarrow \quad q < a,\\
			a' + q' \leq a + q'\quad\Longrightarrow\quad a'\leq a.
		\end{gather*}
	\item For every $K$,
		\begin{equation}
			\label{eq.cancel_at_q'}
			q + q' < \At{K} a + \At{K} q' \quad \Longrightarrow \quad q < \At{K} a.
		\end{equation}
	\end{enumerate}
\end{lemma}

\begin{proof}
	\begin{enumerate}
		\item The assumption of the first claim means that there are $q_1,q_2\in\tQQ$ with $q + q' = q_1 + q_2$ and $q_1 < a$ as well as $q_2 < q'$. Hence $q = q_1 + q_2 - q' < q_1 < a$, as was to be shown. For the second claim, take $q<a'$; then $q+q'<a'+q'\leq a+q'$ which implies $q<a'$ by the first claim.
		\item The assumption is $q + q' < \At{K}(a + q')$ by \cref{lem.at_plus_commute}, or equivalently $\At{K}(q + q' < a + q')$. Therefore $\At{K}(q < a)$ follows from the previous part. \qedhere
	\end{enumerate}
\end{proof}

Roughly, the following says that the lower $\At{K}$-reals coincide with the Dedekind $\At{K}$-reals:

\begin{lemma}
	\label{lem:at_lower_real}
	For all $a : \tLR$,
	\[
		\sup_{\{q \,\mid\, \At{K} q \leq \At{K} a\}} \At{K} q
		\quad = \quad
		\At{K}a 
		\quad = \quad
		\inf_{\{q \, \mid\, \At{K} a \leq \At{K} q\}} \At{K} q.
	\]
\end{lemma}

\begin{proof}
	For the first equality, we show that if $q' : \tQQ$ is such that $q' < \At{K}a$, then there is some $q : \tQQ$ with $\At{K} q \leq \At{K} a$ and $q' < \At{K} q$. For this, we can take any $q$ with $q' < q < \At{K}a$.

	For the second equality, we show that if $q' : \tQQ$ is such that $q' < \At{K} q$ for every $q$ with $\At{K} a \leq \At{K} q$, then also $q' < \At{K}a$. Now $q' < \At{K} q$ is equivalent to $\At{K}(q' < q)$ by definition, and therefore to $\At{K}\bot \:\lor\: (q < q')$ by \cref{lem:rational_compare_at}. Thus we need to show that if $q'$ satisfies $\forall q\ldotp (\At{K} a \leq \At{K} q) \imp (\At{K} \bot \lor (q' < q))$, then also $q' < \At{K} a$. Now by \cref{lem:at_totally_ordered1}, we can assume $\At{K} a \leq \At{K} q'$ without loss of generality; but then the assumption gives $\At{K}\bot \:\lor\: (q' < q')$, or equivalently $\At{K}\bot$, which implies $\At{K}(q' < a)$ as desired.
\end{proof}

\begin{lemma}
	\label{lem.at_plus_rational}
	For $a : \tLR$ and $q : \tQQ$,
	\[
		\At{K}(a + q) = \At{K}a + q.
	\]
\end{lemma}

\begin{proof}
	By \Cref{lem.at_plus_commute} and $q \leq \At{K} q$, it is enough to prove $\At{K}(a + q) \leq \At{K}a + q$. Now if $q' < \At{K}(a + q) = \At{K} a + \At{K}q$, then by \eqref{eq.cancel_at_q'}, we get $q' - q < \At{K} a$, which implies the desired $q' < \At{K} a + q$.
\end{proof}

We also observe that the $\At{K}$-lower reals are totally ordered:

\begin{proposition}
	\label{prop.at_LR_total_order}
	For $a,b : \tLR$,
	\[
		(\At{K}a \leq \At{K}b) \: \lor (\At{K}b \leq \At{K}a).
	\]
\end{proposition}

\begin{proof}
	Let $Q'(q)\coloneqq((q<a)\imp\At{K}(q<b)$, and define $Q(q)\coloneqq\At{K}Q'(q)$ and $P\coloneqq\exists(q:\tQQ)\ldotp Q'(q)\imp\At{K}\bot$. By \cref{ax.atBoolean,prop.N_or_vs_forall}, we have 
	\[
		\left( \exists q : \tQQ \ldotp ((q < a) \imp \At{K} (q < b)) \imp \At{K} \bot \right) \quad \lor \quad \left( \forall q : \tQQ \ldotp \At{K} ((q < a) \imp \At{K} (q < b)) \right).
	\]
	In the second case we have $\At{K}(q<a)\imp\At{K}(q<b)$ for all $q$, i.e.\ $(\At{K}a \leq \At{K}b)$.
	
	In the first case, we have $q$ such that $((q < a) \imp \At{K} (q < b)) \imp \At{K} \bot$. In particular it follows that $\At{K}(q < b) \imp \At{K} \bot$. By \Cref{lem:at_totally_ordered1}, this implies $\At{K} b \leq \At{K} q$ or $\At{K} \bot$; the latter is enough since then either desired inequality follows trivially, so assume $\At{K} b \leq \At{K} q$. Now if $\At{K}(q < a)$, we are done since then we have in particular $\At{K} q \leq \At{K} a$, which combined with the previous inequality gives the desired $\At{K} b \leq \At{K} a$. So assume $(q < a) \imp \At{K} \bot$. But then also $(q < a) \imp \At{K} (q < b)$, and therefore again $\At{K} \bot$, which we have already shown to be enough.
\end{proof}

\begin{proposition}
	\label{prop.at_order_canc}
	The $\At{K}$-lower reals are finitely order-cancellative: if
	\[
		\At{K} a + \At{K} c \leq \At{K} b + \At{K} c
	\]
	and there is $u \in \tQQ$ such that $\At{K} c \leq \At{K} u$, then it follows that $\At{K} a \leq \At{K} b$.
\end{proposition}

Here, the existence of rational $u$ is a finiteness condition on $c$.

\begin{proof}
	By \cref{lem:eps_order}, it is enough to show that $\At{K} a \leq \At{K} b + \varepsilon$ for every rational $\varepsilon > 0$. But by \Cref{lem.at_plus_rational,lem.at_plus_commute}, this inequality is equivalent to $\At{K} a \leq \At{K}(b + \varepsilon)$. By \Cref{prop.at_LR_total_order}, we can assume $\At{K}(b+\varepsilon) \leq \At{K} a$. Combining this with the assumption and using \Cref{lem.at_plus_commute} again, we get
	\[
		\At{K}(b + c + \varepsilon) \leq \At{K}(b + c).
	\]
	Since clearly $\At{K}(b + c) \leq \At{K}(b + c + \varepsilon)$ by $\varepsilon > 0$, we get $\At{K}(b + c + \varepsilon) = \At{K}(b + c)$, and then $\At{K}(b + c + n\varepsilon) = \At{K}(b + c)$ for every $n \in \tNN$ by induction and \Cref{lem.at_plus_commute}.

	Now suppose that we have $q : \tQQ$ with $q < \At{K}a$, and want to show $q < \At{K}b$. By \Cref{lem:at_totally_ordered1}, we can assume $\At{K} b \leq \At{K} q$. Since we already have $\At{K} c \leq \At{K} u$, we get $\At{K}(b + c) \leq \At{K}(q + u)$. Since every lower real has a rational lower bound, we can also find $\ell : \tQQ$ with $\At{K} \ell \leq b + c$.

	Hence we get $\At{K}(\ell + \varepsilon n) \leq \At{K}(q + u)$. Making $n$ large enough gives rationals $q_-\coloneqq q+u$ and $q_+\coloneqq \ell+\varepsilon n$, so $q_- < q_+$. But $\At{K} q_+ \leq \At{K} q_-$ implies $\At{K}(q_- < q_-)$, or $\At{K}\bot$, which is enough.
\end{proof}

\begin{corollary}\label{cor.order_canc}
The lower reals are order-cancellative: if $a+c\leq b+c$ and $c < \infty$, then $a\leq b$.
\end{corollary}

Here, $c < \infty$ means that there is $q : \tQQ$ with $c \leq q$. 

\begin{proof}
	We have $\At{K} a \leq \At{K} b$ for every $K$ by \Cref{prop.at_order_canc}, which is enough by \Cref{prop:enough_points_LR}.
\end{proof}

\begin{lemma}
	\label{lem.at_vs_or_LR}
	For $a : \tLR$ and $P : \Prop$,
	\[
		\At{K}(P \lor a) = \At{K}P \lor \At{K} a.
	\]
\end{lemma}

\begin{proof}
	Evaluate both sides at $q : \tQQ$ and use \Cref{lem.at_vs_or}.
\end{proof}

\begin{corollary}
	\label{cor.P_order_canc}
	Suppose that $P : \Prop$. If
	\[
		P \lor a + P \lor c \: \leq \: P \lor b + P \lor d
	\]
	and $P \lor d \leq P \lor c$, then also $P \lor a \leq P \lor b$.
\end{corollary}

\begin{proof}
	It is enough to consider the case $d = c$.

	We again prove $\At{K}(P \lor a) \leq \At{K}(P \lor b)$ for every $K$, or by \Cref{lem.at_vs_or_LR} equivalently $\At{K} P \lor \At{K} a \leq \At{K} P \lor \At{K} b$. If $\At{K} P$, then there is nothing to prove. If $P \imp \At{K}\bot$, then the disjunction with $\At{K} P$ drops out in all inequalities. Hence in this case, we can apply \Cref{prop.at_order_canc}.
\end{proof}

\section{Other properties of modalities}

\tob{Not sure what we still need from this subsection, but I'm updating it nevertheless}

\begin{lemma}
	\label{lem.prop_to_opens}
	For every $P : \Prop$,
	\[
		P \: \Leftrightarrow \: \forall K \, \exists U : \Prop \ldotp \At{K} U \wedge (U \imp P).
	\]
\end{lemma}

\begin{proof}
From left to right, we know $P$ and can therefore just take $U \coloneqq \top$ for any $K$. From right to left, it is enough to prove $\At{K} P$ for any given $P$ by Axiom~\ref{ax.enough_points}. By $U \imp P$, we can conclude $\At{K} U \imp \At{K} P$ by monotonicity of $\At{K}$. But then we also have the assumption $\At{K} U$, so that the claim $\At{K} P$ follows.
\end{proof}

The following establishes the correspondence between propositions and opens internally:

\begin{proposition}\label{prop:prop_as_opens}
Transposing the canonical map $\At{} : C\BaseSpace \times \Prop \to \Prop$ to $\Prop \to \Prop^{C\BaseSpace}$ identifies $\Prop$ with those predicates $Q : C\BaseSpace \to \Prop$ which satisfy
\[
	\forall K \ldotp \left( \At{K}Q(K) \imp Q(K)\right) \: \land \: \left[ Q(K) \imp \exists U \colon \Prop \ldotp \At{K} U \wedge \forall K' \ldotp \At{K'} U \imp Q(K') \right].
\]
\end{proposition}

\begin{proof}
	Given $P : \Prop$, the associated predicate is $Q \coloneqq \lambda K \ldotp \At{K} P$. Then $\At{K} Q(K) \imp Q(K)$ holds by definition. To see that it has the second required property, we therefore assume $\At{K} P$. Applying \cref{lem.prop_to_opens}, we get $U$ with $\At{K} U$ and $U \imp P$. Now given $K'$ with $\At{K'} U$, we can conclude $\At{K'} P$ by monotonicity of $\At{K'}$.

Clearly if $P, P' : \Prop$ induce the same predicate, then they must be equal, by \cref{ax.enough_points}. Thus it remains to show that every $Q$ which has the required property comes from a proposition.

Given $Q$, we set $P \coloneqq \forall K \ldotp Q(K)$, and claim that $Q = \lambda K \ldotp \At{K} P$. This boils down to showing $Q(K) = \At{K} \forall K' \ldotp Q(K')$ for every $K$. For the right-to-left implication, it is enough to prove $\At{K} Q(K)$ by assumption, but this is clear. So assume $Q(K)$, in order to prove the left-to-right implication. By definition we get $U : \Prop$ with $\At{K} U$ and $\At{K'} U \imp Q(K')$ for every $K'$. So in order to prove $\At{K} \forall K' \ldotp Q(K')$, we can even assume $U$, which implies $\At{K'} U$ and therefore $Q(K')$, as was to be shown.
\end{proof}


\begin{lemma}
	\label{lem.see_vs_exists}
	If $X$ is inhabited, $P : X \to \Prop$ a predicate, and $U : \Prop$, then
	\[
		U \lor \exists x \colon X \ldotp P(x) \quad \Longleftrightarrow \quad \exists x \colon X \ldotp U \lor P(x).
	\]
\end{lemma}

\begin{proof}
The implication from left to right is by case distinction and inhabitedness. The implication from right to left is likewise by cases.
\end{proof}

\begin{lemma}\label{lem.at_vs_codisc}
For every $P : \Prop$ and every $K$,
\[
	\At{K} \See{B} P \: \Leftrightarrow \: \See{B} P \vee \left( \At{K} \See{B} \bot \right).
\]
\end{lemma}

Semantically, both sides of this implication are necessarily true as soon as $K \neq B$.

\begin{proof}
The implication from right to left is a simple case distinction. 

From left to right, we assume $\At{K}(P \vee \See{B} \bot)$, or equivalently $\At{K} P \vee (\At{K} \See{B} \bot)$ by \cref{ax.at_disjunction}. In the case $\At{K} \See{B} \bot$ we are done; in the other case $\At{K} P$, we can assume $P \imp \See{B} \bot$ by $\See{B} = \At{B}$ and the decidability \cref{ax.atBoolean}. But then we have $\At{K} P \imp \At{K} \See{B} \bot$ by monotonicity, from which $\At{K} P$ proves the claim $\At{K} \See{B} \bot$.
\end{proof}

\begin{proposition}\label{prop.flabbiness_equivalence}
Suppose that $X$ is such that
\begin{equation}\label{eq:ourflabby}
	\forall U : \Prop, P : \Prop^X . \left( \In{U} \exists x : X\ldotp P(x) \right) \: \imp \: \exists x : X \ldotp \In{U} P(x).
\end{equation}
Then $X$ is flabby. Conversely, if $X$ is flabby and satisfies choice, then \cref{eq:ourflabby} holds.
\end{proposition}

\todo[inline]{If necessary, we can probably weaken ``satisfies choice'' to ``satisfies supports split (internally)''}

\begin{proof}
For the first claim, suppose that $P : X \to \Prop$ is subsingleton. Then applying \cref{eq:ourflabby} with $U \coloneqq \exists x : X \ldotp P(x)$ directly results in the claim $\exists x : X\ldotp (\exists y : X\ldotp P(y)) \imp P(x)$.

Conversely, let $X$ be flabby and satisfy choice, and let $P$ with $\In{U} \exists x : X. P(x)$ be given. Then the subobject $\{ x \mid P(x) \}$ still satisfies choice~\cite[Lemma D.4.5.9]{Johnstone:2002a}. Hence the eqic part of the image factorization $\{ x \mid P(x) \} \twoheadrightarrow \im(!) \hookrightarrow 1$ splits, and we obtain a subsingleton $S : X\to\Prop$ satisfying $(\exists y : X\ldotp P(y)) \imp \exists x : X\ldotp S(x)$. Applying the flabbiness assumption to $S$ gives $x : X$ with $(\exists y : X\ldotp S(y)) \imp S(x)$. Using this $x$ in the consequent of \cref{eq:ourflabby} works, since we now know
\[
	(\exists y : X\ldotp P(y)) \imp P(x),
\]
which remains valid after applying $\In{U}$ on both sides.
\end{proof}

\begin{example}
In our old example of $\Shv(C\{a,b\})$, the object $X \coloneqq \Prop$ does not satisfy \eqref{eq:ourflabby}, as one can see by taking $U \coloneqq \{a,b\}$ and
\[
	P(x) := ((\{a\}\imp x) \vee (\{b\}\imp x)) \wedge \neg x.
\]
Then $C\{a,b\} \models \In{U} \exists x. P(x)$ since $\{a\} \models \neg \{b\}$ and $\{b\} \models \neg \{a\}$, but $C\{a,b\} \not\models \exists x\ldotp \In{U} P(x)$, since none of the choices for $x$ works.
\end{example}

Recall the semantics of $\At{K}$: we have $U \models \At{K} P$ if and only if $K \not\in W$, or there is $V\subseteq U$ with $K\in V$ and $W \models P$. The semantics of $\In{U}$ is that $V \models \In{U} P$ if and only if $U \cap V \models P$.

\begin{axiom}\label{axiom:completely_prime}
For every $P : \Prop \to \Prop$,
\[
	\At{K} \exists U\ldotp P(U) \wedge U \quad \imp \quad \exists U\ldotp (\In{U} P(U)) \wedge \At{K} U.
\]
\end{axiom}

Roughly speaking, this expresses the fact that a point $K$ is a completely prime filter of opens in $C\BaseSpace$: if $K\in \bigvee_{U : P(U)} U$, then there is $U$ with $P(U)$ and $K\in U$.

\begin{soundproof}
    Suppose $W \models @_K \exists U. P(U) \wedge U$. Then either $K$ is not in $W$ or $K$ is in $W$.

    If $K$ is not in $W$, then the claim follows by covering $W$ by itself, and $W \models (\In{\bot} P(\bot)) \wedge \At{K} \bot$, where both $W \models \In{\bot} P(\bot)$ and $W \models \At{K} \bot$ hold trivially.

    Now assume $K$ is in $W$. Then the assumption says that there is some $W'\subseteq W$ with $K\in W'$ and $W' \models \exists U\ldotp P(U) \wedge U$. This means that there is a cover $(V_i\to W')$ together with $U_i\in\Prop(V_i)$ such that $V_i \models P(U_i) \wedge U_i$. But now we have $U_i \subseteq V_i$ as well as $V_i \subseteq U_i$, so that $U_i = V_i$, and therefore $V_i \models P(V_i)$. Since $V_i$ is a cover of $W'$, there exists some $k$ such that $K \in V_k$. We now show the claim $W \models \exists U. (\In{U} P(U)) \wedge \At{K} U$ by proving $W \models (\In{V_k} P(V_k)) \wedge \At{K} V_k$, corresponding to the trivial cover of $W$ by itself. But now $W \models \In{V_k} P(V_k)$ follows from $V_k \models P(V_k)$, and $W \models \At{K} V_k$ from $K\in V_k$.
\end{soundproof}

\chapter{Internal valuations}

\section{Definition}

\begin{definition}
	Let $(\mathcal{P},\leq)$ be an internal poset. A subtype $D:\mathcal{P}\to\Prop$ is called \emph{directed} if it satisfies the following:
\begin{enumerate}
	\item $\exists(x:\mathcal{P})\ldotp Dx$.
	\item $\forall(x,y:\subtype{D})\ldotp\exists(z:\subtype{D})\ldotp(x\leq z)\wedge (y\leq z)$.
\end{enumerate}
Denote the conjunction of these two conditions as $\const{Drct}:(\mathcal{P}\to\Prop)\to\Prop$.
\end{definition}



\begin{definition}\label{def.valuation}
Suppose $(X,\Op(X))$ is an internal space. A \emph{valuation} on $X$ is a function $\mu:\Op(X)\to\tLR$ satisfying the following:
\begin{enumerate}
	\item $\mu(\varnothing)=0$.
	\item \emph{Normalization}: $\mu(X) = 1$.
	\item \emph{Monotonicity}: for all $U,V:\Op(X)$,
		\[
			\left( U \imp V \right) \: \Longrightarrow \: (\mu(U)\leq\mu(V)).
		\]
	\item \emph{Inclusion-exclusion} or \emph{modularity}: for all $U,V:\Op(X)$,
		\[
			\mu(U)+\mu(V)=\mu(U\vee V)+\mu(U\wedge V).
		\]
	\item \emph{Scott continuity}: for all directed $D:\Op(X)\to\Prop$, the supremum is preserved by $\mu$, in the sense that for all $q : \tQQ$ with $q \geq 0$,
		\[
			\mu(\exists U : \Op(X) \ldotp D(U) \land U)(q) \: = \: \exists U : \Op(X) \ldotp D(U) \land \mu(U)(q).
		\]
\end{enumerate}
\end{definition}

While such a function would more properly be called \emph{normalized continuous valuation}, we shorten this to \emph{valuation} for convenience, as this will be the only kind of valuation that we will work with.

\section{Valuations on $\Prop$}

\tob{Perhaps better: say that we are talking about valuations on the singleton space, as per \Cref{def.valuation}}

Consider the function $i\colon\Prop\to\tLR$ defined as follows
\[i(P)(q)\coloneqq(q<0)\vee((q<1)\wedge P).\]
We next show that it is a probability valuation (probability valuation). Later we will show that, under some conditions, it is unique such.

\begin{lemma}
The function $i\colon\Prop\to\tLR$ defined above is indeed a probability valuation on $\Prop$.
\end{lemma}

\begin{proof}
We first show that $i$ is a probability valuation. The conditions $i(\bot) = 0$, monotonicity, and $i(\top) = 1$ are trivial. We prove the inclusion-exclusion relation
\[
	i(P) + i(Q) = i(P \land Q) + i(P \lor Q)
\]
by proving both inequalities. For $\geq$, we need to prove that if we have rationals $q_\land < i(P \land Q)$ and $q_\lor < i(P \lor Q)$, then we can find rationals $q_P$ and $q_Q$ with $q_P + q_Q \leq q_\land + q_\lor$ and $q_P < i(P)$ and $q_Q < i(Q)$. So we have the assumptions
\begin{equation}
	\label{eq.q_rhs}
	(q_\land < 0) \lor ( (q_\land < 1) \land P \land Q), \qquad (q_\lor < 0) \lor ( (q_\lor < 1) \land (P \lor Q)).
\end{equation}
In case that $(q_\land < 1) \land P \land Q$ holds, we can simply take $q_P := q_\land$ and $q_Q := q_\lor$ and be done. So assume $q_\land < 0$. Then if $(q_\lor < 1) \land (P \lor Q)$ holds, assume $P$ without loss of generality. Then similarly $q_P := q_\lor$ and $q_Q := q_\land$ works. Thus we assume $q_\lor < 0$ as well; but then again $q_P := q_\land$ and $q_Q := q_\lor$ works.

For the other inequality, we start with $q_P$ and $q_Q$ such that
\[
	(q_P < 0) \lor ( (q_P < 1) \land P), \qquad (q_Q < 0) \lor ( (q_Q < 1) \land Q),
\]
and put $q_\land := \min(q_P, q_Q)$ and $q_\lor := \max(q_P, q_Q)$, so that $q_\land + q_\lor = q_P + q_Q$ is guaranteed. Then~\eqref{eq.q_rhs} follows.

In order to show that $i$ preserves directed suprema, let $D : \Prop \to \Prop$ be directed. Then
\[
	\sup(D) = \exists P : \Prop \ldotp D(P) \land P,
\]
and what we need to show is that for all rational $q$, we have
\[
	(q < 0) \lor ( (q < 1) \land \exists P : \Prop \ldotp D(P) \land P) \qquad \Longleftrightarrow \qquad \exists P : \Prop \ldotp D(P) \land \left[ (q < 0) \lor ( (q < 1) \land P ) \right],
\]
where we have used the fact that suprema of lower reals are computed ``at every $q$''. The implication from left to right is clear in the second case, and in case $q < 0$ uses inhabitedness of $D$. The implication from right to left is by a straightforward case analysis. Hence $i$ preserves directed suprema.
\end{proof}

\begin{proposition}
\label{prop.sandwich}
For any probability valuation $\mu\colon\Prop\to\tLR$, we have $i\leq\mu\leq i\neg\neg$.
\end{proposition}

\begin{proof}
To show $i\leq\mu$, we need to prove that
\[
	(q < 0) \lor ( (q < 1) \land P) \quad \Longrightarrow \quad \mu(P)(q)
\]
for every $P : \Prop$. Since $\mu(P) \geq 0$, this clearly holds in the case $q < 0$, so that we can assume $q < 1$ and $P$. But then $P = \top$, and therefore $\mu(P) = 1$, which implies the claim.

To show that $\mu\leq i\neg\neg$, take $P:\Prop$ and $q:\tQQ$ with $0\leq q < 1$; we need to show $\mu(P)(q)\imp\neg\neg P$. Assume $\mu(P)(q)$ and $\neg P$. Then by monotonicity applied to $P \imp \bot$ we have $\mu(\bot)(q)$, and by $\mu(\bot)=0$, this means $q<0$, a contradiction.
\end{proof}

\begin{lemma}
\label{lem.regularly_based}
For all $P : \Prop$,
\[
	P \quad \Longleftrightarrow \quad \exists(Q:\Prop)\ldotp (Q\imp P)\wedge(\neg\neg Q\imp Q)\wedge Q.
\]
\end{lemma}

\begin{proof}
	The implication from right to left is trivial. From left to right, take $Q := \top$.
\end{proof}

\begin{lemma}
	\label{lem.sup_vs_directed}
	Let $L$ be a complete lattice and $S\subseteq L$ a subtype with $0\in S$, closed under binary joins and under directed joins. Then $S$ is closed under all joins.
\end{lemma}

For $n : \tNN$, we also write $[n] := \{1,\ldots,n\}$ for the subtype of positive natural numbers less than or equal to $n$.

\begin{proof}
	First, by induction on $n : \tNN$, we see that every (internally) finite join $\ell_1 \lor \ldots \lor \ell_n$ is again in $S$.

	Second, let $P : L \to \Prop$ be a subpredicate of $S$, meaning that $P(\ell) \imp (\ell \in S)$; we will show that $\sup P \in S$. To this end, consider the new predicate
	\[
		Q(\ell) := \exists n : \tNN, j : [n] \to S \ldotp \left( \ell = j(1) \lor \ldots \lor j(n) \right) \land \forall k \in [n]\ldotp P(j(k)).
	\]
	Then it is easy to see that $Q$ is directed. Moreover, we show that $\sup P = \sup Q$. For if $\hat{\ell} \in L$ is such that $P(\ell) \imp (\ell \leq \hat{\ell})$, then it is easy to see that $Q(\ell) \imp (\ell \leq \hat{\ell})$ as well, and conversely.

	Hence it is enough to show that $S$ contains $\sup Q$. But since $Q$ is directed, it is enough to show that $Q(\ell) \imp (\ell \in S)$. But the assumption $Q(\ell)$ implies $\ell = j(1) \lor \ldots \lor j(n)$ for suitable $n$ and $j$ with $P(j(k))$, and therefore also $j(k) \in S$. Hence $\ell \in S$, as was to be shown.
\end{proof}

\begin{proposition}\label{prop.unique_probability_valuation_on_prop}
$i\colon\Prop\to\tLR$ is the unique probability valuation on $\Prop$. 
\end{proposition}

\begin{proof}
Let $\mu$ be an arbitrary valuation; we want to show $\mu=i$. Define a subtype $T\coloneqq\{P:\Prop\mid\mu(P)=i(P)\}$, where $T$ stands for target; we want to show that $T=\Prop$. If $Q = \neg\neg Q$, then by \cref{prop.sandwich} we have $i(Q)\leq\mu(Q)\leq i(\neg\neg Q)=i(Q)$, so that $Q\in T$.

Hence by \Cref{lem.sup_vs_directed} together with the supremum of \Cref{lem.regularly_based} and the fact that both $\mu$ and $i$ preserve directed suprema, it is enough to show that $T$ is closed under binary disjunctions. So suppose that we are given $Q_1,Q_2\in T$. Then
\begin{align*}
	i(Q_1\vee Q_2)+i(Q_1\wedge Q_2)
	&\leq\mu(Q_1\vee Q_2)+i(Q_1\wedge Q_2)\\
	&\leq \mu(Q_1\vee Q_2)+\mu(Q_1\wedge Q_2)\\
	&=\mu(Q_1)+\mu(Q_2)\\
	&=i(Q_1)+i(Q_2).
\end{align*}
	Since $i(Q_1\vee Q_2)+i(Q_1\wedge Q_2)=i(Q_1)+i(Q_2)$, it follows that all the above inequalities must be equalities. By the cancellativity of \Cref{cor.order_canc}, this results in particular in $\mu(Q_1\vee Q_2)=i(Q_1\vee Q_2)$.
\end{proof}



\section{Interaction of valuations with modalities}

From now on, we consider valuations on an internal topological space $(X,\Op(X))$.

The following holds for any open modality $\In{U}\coloneqq (U\imp -)$.

% T: The following was problematic, since there's no reason for why $\In{U}\mu(P)$ would even again be a lower real
%\begin{proposition}\label{prop.In_probability_valuation}
%	\[\In{U}\mu(\In{U}P)=\In{U}\mu(P).\]
%\end{proposition}
%
%Semantically, this is just naturality of $\mu$ as a morphism of sheaves.

\begin{proof}
	One direction is obvious, so suppose $U\imp\mu(U\imp P)$; we want to show $U\imp\mu(P)$. But assuming $U$ we have $\mu(U\imp P)$ which is the same as $\mu(P)$.
\end{proof}

\begin{proposition}\label{prop.valuations_OR}
	Suppose $P:\Prop$ and $Q:\Op(X)$. Then
	\begin{enumerate}
		\item \[\mu(P\vee Q)= \max(i(P),\mu(Q)). \]
		\item \[\mu(P\wedge Q) = \min(i(P),\mu(Q)).\]
	\end{enumerate}
\end{proposition} 

Here, the left-hand sides are defined thanks to the openness of the constant predicate $P$, per \Cref{constants_open}.

\begin{proof}
	\begin{enumerate}
	\item By inclusion-exclusion, we have $\mu(P\vee Q)+\mu(P\wedge Q)=\mu(P)+\mu(Q)$. By \Cref{lem.or_add_LR}, we get upon applying $P \lor -$ to both sides,
		\[
			P \lor \mu(P \vee Q) + P \lor \mu (P\wedge Q) = P \lor \mu(P) + P \lor \mu(Q).
		\]
		On the other hand, we have $P \lor \mu(P) \leq P \lor \mu(P \land Q)$, which follows from $\mu(P) = i(P)$ by \Cref{prop.unique_probability_valuation_on_prop}. Hence the cancellativity of \Cref{cor.P_order_canc} gives $P \lor \mu(P \vee Q) = P \lor \mu(Q)$. In particular, $\mu(P \vee Q) \leq P \lor \mu(Q)$, which together with $\mu(P \vee Q) \leq 1$ shows the $\leq$ inequality. The other one is trivial.
	
	\item Using inclusion-exclusion, \cref{prop.unique_probability_valuation_on_prop}, and the fact that $\min(a,b)+\max(a,b)=a+b$ for any lower reals $a,b:\tLR$ (\Cref{lem.max_min_LR}), we have
\begin{align*}
	\mu(P\wedge Q)+\mu(P\vee Q) & =\mu(P)+\mu(Q) \\
	&=\max(i(P),\mu(Q))+\min(i(P),\mu(Q))\\
	&=\mu(P\vee Q)+\min(i(P),\mu(Q)).
\end{align*}
Thus we are done by the first part and the cancellativity of \cref{cor.order_canc}. \qedhere
\end{enumerate}
\end{proof}

\begin{lemma}
	\label{lem.naturality}
	Suppose that $(X,\Op(X))$ is an internal space and $\mu : \Op(X) \to \tLR$ a valuation. Then for any $P, Q : \Op(X)$ with $\At{K} P$,
	\[
		\At{K} \mu(Q) = \At{K} \mu(P \imp Q).
	\]
\end{lemma}

\begin{proof}
  Assuming $P$, the statement $\mu(Q)=\mu(P\imp Q)$ is a tautology. Thus we have $P\imp(\mu(Q)=\mu(P\imp Q))$, so by the naturality of $\At{K}$ we have
  \[\At{K}P\imp\At{K}\big(\mu(Q)=\mu(P\imp Q)\big)\]
  The result follows by another application of naturality.
%	The inequality $\At{K} \mu(Q) \leq \At{K} \mu(P \imp Q)$ is clear by monotonicity. For the other direction, we prove that
%	\[
%		\At{K} \mu(P \imp Q)(q) \quad \Longrightarrow \quad \At{K} \mu(Q)(q)
%	\]
%	for every $q : \tQQ$. Since we want to prove an $\At{K}$-closed statement, we can strengthen the assumption of $\At{K} P$ to $P$. But then $P \imp Q$ is already equal to $Q$.
\end{proof}

\begin{lemma}
	\label{lem.mu_at_Q}
	For every $Q : \Op(X)$ which satisfies $\At{K} Q$, we have
	\[
		\At{K} \mu(Q) = \At{K} 1.
	\]
\end{lemma}

\begin{proof}
	Apply \Cref{lem.naturality} with $P := Q$.
\end{proof}

\begin{theorem}
	For every internal space $(X,\Op(X))$ and valuation $\mu : \Op(X) \to \tLR$, we have
	\[
		\At{K} \mu(P) = \At{K} \mu(\At{K} P).
	\]
	for every $P : \Op(X)$ and $K : \CB$.
\end{theorem}

\begin{proof}
	As instances of the inclusion-exclusion relation, we have
	\begin{align*}
		\mu(P) + \mu(P \imp \At{K} \bot) & = \mu(P \lor (P \imp \At{K} \bot)) +  \mu(P \land (P \imp \At{K} \bot),\\
		\mu(\At{K}P) + \mu(P \imp \At{K} \bot) & = \mu(\At{K} P \lor (P \imp \At{K} \bot)) + \mu(P \land (\At{K} P \imp \At{K} \bot),
	\end{align*}
	Using \cref{lem.at_plus_commute}, we can hit both equations with $\At{K}$, and we claim that all terms in the two equations match; while we are interested in proving this for the first terms, we do so by showing that it holds for the other three pairs. The nontrivial cases are the two pairs on the right-hand sides.
	
	For the first pair on the right-hand sides, we use \Cref{lem.mu_at_Q} together with the fact that hitting the two arguments of $\mu$ with $\At{K}$ gives $\top$, since $\At{K} P \lor (P \imp \At{K})$. For the second pair on the right-hand sides, we use the fact that they both imply $\At{K} \bot$, so that they are both upper bounded by $\At{K} \mu(\At{K} \bot) = \At{K} 0$ by monotonicity. But since $\mu$ is nonnegative to begin with, this value is clearly also a lower bounded, and hence both terms evaluate to $\At{K} 0$.
\end{proof}

\chapter{Families of point valuations}

For $\Op(X)$ the type of opens of an internal space, we write $\At{K} \Op(X)$ for the subtype of $\At{K}$-closed opens. As a special case, we have $\At{K} \Prop$, the type of $\At{K}$-closed propositions. 

\begin{definition}
	An \emph{$\At{K}$-topology} on $X$ is a function $\Op:(X\to\At{K}\Prop)\to\At{K}\Prop$ satisfying the following conditions:
\begin{enumerate}
	\item $\Op(\lambda(x:X)\ldotp\top)$.
	\item $\forall(\phi_1,\phi_2:X\to\At{K}\Prop)\ldotp(\Op(\phi_1)\wedge\Op(\phi_2))\imp\Op(\lambda(x:X)\ldotp\phi_1 x\wedge\phi_2 x)$.
	\item Suppose given $I: \Op(X) \to \At{K}\Prop$. Then $\Op(\lambda(x:X)\ldotp\At{K}\exists(\phi:X\to\At{K}\Prop)\ldotp I(\phi)\wedge\phi (x))$.
\end{enumerate}
\end{definition}

The idea is that semantically, an $\At{K}$-topology only equips the stalk of $X$ at $K$ with a topology.

\begin{lemma}
	For $(X, \Op(X))$ an internal space, the type $\At{K} \Op(X)$ is an $\At{K}$-topology on $X$.
\end{lemma}

\begin{proof}
	The binary intersection property holds because it holds for $\Op(X)$.

	Concerning arbitrary unions, we use the flabbiness of $\Op(X)$ of \Cref{prop:opens_flabby} and \Cref{ax:flabby_vs_at}. This shows that the predicate $\lambda x. \exists \phi : X \to \At{K} \Prop \ldotp I(\phi) \land \phi(x)$ is again $\At{K}$-closed, since both $I(\phi)$ and $\phi(x)$ are. Since this predicate is again in $\Op(X)$, the claim follows by definition of $\At{K} \Op(X)$.
\end{proof}

We also write $\At{K} \tLR$ for the type of $\At{K}$-closed lower reals, which is isomorphic to the type of $\At{K}$-lower reals of TTT.

\begin{definition}
	Let $(X, \Op(X))$ be an internal space. An \emph{$\At{K}$-valuation} is a map $\nu_K : \At{K} \Op(X) \to \At{K} \tLR$
	\tob{Think about directedness}
\end{definition}





\chapter{Semantics of valuations on spaces of compacts}

While the results of the previous sections were completely internal, we now use them in order to analyze the semantics of probability valuation's in $\Shv(C\BaseSpace)$.

An internal probability valuation $\mu\colon\Prop^X\to\tLR$ on $\cat{E}$ gives rise to an external monotonic map
\[|\mu|\colon\cat{E}(X,\Omega)\times\pt(\cat{E})\to\LR\]
that is Scott continuous (at least in each variable) and a probability valuation for each point $a\in\pt(\cat{E})$.

\todo[inline]{I don't see this at this level of generality. Are we trying to transport the valuation along $a^*$ here?}

Consider the case where $\BaseSpace$ is a compact space, $C\BaseSpace$ is its space of compacts, and $\cat{E}=\Shv(C\BaseSpace)$. In this case, if $X\in\cat{E}$ is a sheaf, then evaluation at $\BaseSpace$ is the same thing as taking global sections, $X(\BaseSpace)\cong\Gamma(X)$; we denote it by $|X|$. If $r\in\tLR$ is a lower real, then we also write $|r|$ for the associated lower semi-continuous function on $C\BaseSpace$. For example, we have $|\mu(\top)|(K) = 1$ by normalization for every $K\in C\BaseSpace$.

\begin{lemma}\label{lemma.inequality_rand9865}
Let $\cat{E}$ be as above. For any point $K\in\pt(\cat{E})$ and predicate $P\colon X\to \Prop$, we have
\[
\sup_{U\ni K} \At{\BaseSpace} \mu(\At{\BaseSpace}\In{U}P) \leq \At{\BaseSpace} \At{K} \mu(P),
\]
where $U$ ranges over all (basic) neighborhoods of $K$.
\end{lemma}
\begin{proof}
The proof uses a few facts:
\begin{enumerate}
	\item The fact that $\BaseSpace$ is Scott-minimal is internally expressed as an equivalence of modalities, $\At{\BaseSpace}=\See{\BaseSpace}$.
	\item $\See{\BaseSpace}\mu(P)=\mu(\See{\BaseSpace}P)$ by \cref{prop.valuations_OR}
	\item If $K\in U$, then $\At{K}\In{U} = \At{K}$.
\end{enumerate}
We have now established the following:
\begin{align}
\nonumber
	\sup_{U\ni K}\mu(\At{\BaseSpace}\In{U}P)
	&=\sup_{U\ni K}\mu(\At{\BaseSpace}\In{U}P)&\text{\cref{prop.double_at}} \\\nonumber
	&=\sup_{U\ni K}\mu(\See{\BaseSpace}\In{U}P)&\text{item 1}\\\nonumber
	&=\sup_{U\ni K}\See{\BaseSpace}\mu(\In{U}P)&\text{item 2}\\\nonumber
	&=\sup_{U\ni K}\At{\BaseSpace}\mu(\In{U}P)&\text{trivial}\\\nonumber
	&\leq\sup_{U\ni K}\At{\BaseSpace}\At{K}\mu(\In{U}P)&\text{increasing}\\\nonumber
	&=\sup_{U\ni K}\At{\BaseSpace}\At{K}\In{U}\mu(\In{U}P)&\text{item 3}\\\nonumber
	&=\sup_{U\ni K}\At{\BaseSpace}\At{K}\In{U}\mu(P)&\text{\cref{prop.In_probability_valuation}}\\\nonumber
	&=\At{\BaseSpace}\At{K}\mu(P)&\text{item 3}
\end{align}
\end{proof}


We are ready to prove that this inequality is actually an equality. This shows that we can recover the action of $\mu$ on arbitrary (global) predicates from its action on global sections of the very special $\At{B}$ form.

\begin{theorem}\label{thm.discrete_internal_space}
Let $\BaseSpace$ be a compact Hausdorff space, $C\BaseSpace$ is its space of compacts, $X\in \Shv(C\BaseSpace)$ and $\mu : \Prop^X \to \tLR$ a probability valuation. For any predicate $P:X\to\Prop$ and $K\in C\BaseSpace$, we have
\begin{equation}\label{eq:mu_formula}
\At{B} \At{K} \mu(P) = \sup_{U\ni K} \At{B} \mu(\At{B}\In{U}P).
\end{equation}
\end{theorem}
\begin{proof}
Using \cref{lemma.inequality_rand9865} and inserting a constant sup, we have the inequality
\[
	\At{\BaseSpace}\At{K} \sup_{U\ni K} \mu(P) \geq \sup_{U\ni K} \At{\BaseSpace} \mu(\At{\BaseSpace}\In{U}P) .
\]
Thanks to \cref{lem.see_vs_exists}, the modality $\At{B} = \See{B}$ on the right-hand side commutes with the supremum. Thus by Scott continuity of $\mu$, we have equivalently
\[
	\At{\BaseSpace} \At{K} \mu(\exists U\ldotp \At{K}U\wedge P) \geq \At{\BaseSpace} \mu(\exists U\ldotp \At{K}U\wedge \At{B}\In{U}P) .
\]
We now make the left-hand side (non-strictly) larger by replacing $P$ by $\In{U}P$, and similarly the right-hand side smaller by dropping one $\At{B}$ modality, resulting in
\[
	\At{\BaseSpace} \At{K} \mu(\exists U\ldotp \At{K}U\wedge\In{U}P) \geq \At{\BaseSpace} \mu(\exists U\ldotp \At{K}U\wedge\In{U}P) .
\]
In the following, we will prove that this inequality is actually an inequality. This forces the previous inequalities to hold with equality as well, finishing off the proof.

Now we use one of our favorite tricks: we apply the above inequality to both $P$ and $P\imp\At{a}\bot$, and then add them and use inclusion-exclusion.
We obtain something of the form $\At{\BaseSpace}\mu(X) + \At{\BaseSpace}\mu(Y) \leq \At{\BaseSpace}\At{K} \mu(X) + \At{\BaseSpace}\At{K} \mu(Y)$, where
\begin{align*}
	X\coloneqq(\exists U\ldotp \At{K}U\wedge\In{U}P)\wedge\big(\exists U\ldotp \At{K}U\wedge\In{U}(P\imp\At{K}\bot)\big), \\
	Y\coloneqq(\exists U\ldotp \At{K}U\wedge\In{U}P)\vee\big(\exists U\ldotp \At{K}U\wedge\In{U}(P\imp\At{K}\bot)\big).	
\end{align*}
We will show that $Y=\top$ and that $X\imp\At{K}\bot$. This will complete the proof because $\At{K}\mu(\At{K}\bot) = \At{B}\mu(\At{K}\bot) = 0$.

First we show $X\imp\At{K}\bot$. We claim $X\imp\exists U\ldotp \At{K}U\wedge\In{U}(P\wedge P\imp\At{K}\bot)$; indeed, take $U$ to be the conjunction of the existentials from $X$ and use the fact that $\At{K}$ and $\In{U}$ commute with $\wedge$. From here we get $\exists U\ldotp\At{K}U\wedge\In{U}\At{K}\bot$. Since $\In{U}P$ means $U\imp P$, we obtain $\At{K}\bot$ by monotonicity of $\In{U}$ and idempotence of $\At{K}$.

We now show $Y=\top$. By cases we have $\exists U\ldotp\At{K}U\wedge(\In{U}P\vee\In{U}(P\imp\At{K}\bot))\imp Y$. Thus to prove $Y$, it is enough to show that the following existential statement holds:
\[\exists U\ldotp\At{K}U\wedge(\In{U}P\vee(P\imp\At{K}\bot)).\]
By \cref{AtBoolean}, we have $\At{K}P\vee (P\imp\At{K}\bot)$. In the first case, take $U=P$ and the existential holds. In the second case, take $U$ to be any open for which $\At{K}U$ holds. This completes the proof.
\end{proof}


\section{Main theorem: correspondence with external valuations}\label{sec.main_thm}

We have a sequence of adjoints:
\[
\begin{tikzcd}[column sep=100pt]
	\Shv(\CB)
		\ar[r, shift left=40pt, "\Pi_0" description]
		\ar[r, shift left=0pt, "\Gamma" description]
		\ar[r, shift left=30pt, phantom, "\scriptstyle\bot"]
		\ar[r, shift left=10pt, phantom, "\scriptstyle\bot"]
		\ar[r, shift right=10pt, phantom, "\scriptstyle\bot"]
	&
	\smset.
		\ar[l, shift right=20pt, "\Const" description]
		\ar[l, shift left=20pt, "\Sky_B" description]
\end{tikzcd}
\]
where $\Const$ and $\Sky_B$ are fully faithful; this implies that the unit map for the $\Gamma\vdash\Const$ adjunction and the counit map for the $\Gamma\dashv\Sky_B$ adjunctions are isomorphisms:
\[
	S\cong\Gamma\circ\Const(S)
	\qquad\text{and}\qquad
	\Gamma\circ\Sky_B(S)\cong S
\]
for any set $S$. Thus the structure isomorphisms for these adjunctions are, in one direction, given by applying $\Gamma$ and composing with the above unit and counit isomorphisms respectively. We denote their inverses by $\alpha$ and $\beta$:
\begin{align*}
	\alpha\colon\Hom_\smset(S,\Gamma(X))&\To{\cong}\Hom_{\Shv(\CB)}(\Const(S),X)\\
	\beta\colon\Hom_\smset(\Gamma(X),S)&\To{\cong}\Hom_{\Shv(\CB)}(X,\Sky_B(S))
\end{align*}

We can identify $\asSh_{\AtInline{B}}\Prop$ with $\Sky_B(\Bool)$, so we have a sheafification map $\toSh_{\AtInline{B}}\colon\Prop\to\Sky_B(\Bool)$. We will also use the function $\classify{\Sky_B(\top)}\colon\Sky_B(\Bool)\to\Prop$ classifying the sheaf morphism corresponding (via the functor $\Sky_B$) to the function $\top\colon 1\to\Bool$.

In \cref{eqn.mu_to_muprime}, we propose a function
\begin{multline*}
  \{\tn{internal probability valuations } \mu\colon \Prop^X \to\tLR\}
  \to\\
  \{\tn{external probability valuations } \mu'\colon \Bool^{X(B)} \to\LR\}
\end{multline*}
Given an internal $\mu\colon\Prop^X\to\tLR$, define $\mu'\colon\Bool^{\Gamma(X)}\to\LR$ on $P'\colon\Gamma(X)\to\Bool$ and $q\in\QQ$ to be
\begin{equation}\label{eqn.mu_to_muprime}
	\mu'(P')(q)\coloneqq 
	\Gamma\left(\toSh_{\AtInline{B}}\left(\mu\left(\classify{\Sky_B(\top)}\circ\beta(P')\right)(q)\right)\right).
\end{equation}
We still need to check that the $\mu'(P')$ is a lower real, and that $\mu'$ is an external valuation; we also have to produce an inverse. Before doing that, we parse the formula \eqref{eqn.mu_to_muprime} in the following diagram:
\[
\begin{tikzpicture}
	\begin{scope}[font=\scriptsize]
		\node (X) {$\mu\big(X$};
		\node[right=1 of X] (SB1) {$\Sky_B(\Bool)$};
		\node[right=1.5 of SB1] (Prop1) {$\Prop\big)$};
		\draw[->] (X) to node[above, font=\tiny] {$\beta(P')$} (SB1);
		\draw[->] (SB1) to node[above, font=\tiny] {$\classify{\Sky_B(\top)}$} (Prop1);
	\end{scope}
	\begin{scope}[font=\footnotesize]
		\node[below left=-.3 and 0 of X] (Q1) {$\Gamma\big(\tQQ$};
		\node[below right=-.3 and 0 of Prop1] (Prop2) {$\Prop$};
		\node[right=1 of Prop2] (SB2) {$\Prop_{\AtInline{B}}=\Sky_B(\Bool)\big)$};
		\draw[->] (Q1) -- (Prop2);
		\draw[->] (Prop2) to node[above, font=\scriptsize] {$\toSh_{\AtInline{B}}$} (SB2);
	\end{scope}
	\begin{scope}[font=\normalsize]
		\node[below left=-.3 and 0 of Q1] (Q2) {$\QQ$};
		\node[below right=-.3 and 0 of SB2] (Bool) {$\Bool$};
		\draw[->] (Q2) -- (Bool);
	\end{scope}
\end{tikzpicture}
\]

We will define an inverse in \cref{eqn.muprime_to_mu}. In order to do so, we need some other definitions. Let $\Pt$ be the constant type of points in $\CB$, and let $\cc\colon\Pt\to\Prop$ be an internalization of the ``closure-complement'' map, i.e.\ the sheaf morphism $\Pt=\Const(|\CB|)\to\church{\Prop}$ whose mate is the function $|\CB|\to\Gamma(\Prop)$ sending each point $K\in|\CB|$ to the open set $(\down K)^c$ given by the complement of the closure of the point $K$. We take $\Pt$ and $\CB$ to be an atomic types and atomic term, if need be. From this point, we can define
\[\At{}\colon\Pt\to\Prop\to\Prop,\qquad\At{K}P\coloneqq (P\imp\cc(K))\imp\cc(K)\]
and we take $\forall(K:\Pt)(P:\Prop)\ldotp\At{K}P\vee (P\imp\At{K}\bot)$ as an axiom. If $\CB$ has enough points, we also take as an axiom
\[\forall(P:\Prop)\ldotp P\iff\forall(K:\Pt)\ldotp\At{K}P.\]

We are ready to define the inverse to \cref{eqn.mu_to_muprime}. Suppose given $\mu'$. We define $\mu\colon\Prop^X\to\tLR$ on a predicate $P\colon X\to\Prop$ and a rational $q:\tQQ$ to be
\begin{multline}\label{eqn.muprime_to_mu}
  \mu(P)(q)\coloneqq\forall(K:\Pt)\ldotp\exists(U:\Prop)\ldotp\At{K}U\wedge\\
  \classify{\Sky_{K}(\top)}\left(\Sky_{K}\left(\mu'\left(\Gamma(\toSh_{\AtInline{B}}(\In{U}P))\right)(q)\right)\right)
\end{multline}
We parse \eqref{eqn.muprime_to_mu} in the following diagram, again using the identification $\Prop_{\AtInline{B}}=\Sky_B(\Bool)$:
\[
\begin{tikzpicture}[x=.8cm]
	\begin{scope}[font=\scriptsize]
		\node (X) {$\mu'\big(\Gamma\big(X$};
		\node[right=.8 of X] (SB1) {$\Prop$};
		\node[right=.8 of SB1] (Prop1) {$\Sky_B(\Bool)\big)\big)$};
		\draw[->] (X) to node[above, font=\tiny] {$\In{U}P$} (SB1);
		\draw[->] (SB1) to node[above, font=\tiny] {$\toSh_{\AtInline{B}}$} (Prop1);
	\end{scope}
	\begin{scope}[font=\footnotesize]
		\node[below left=-.3 and 0 of X] (Q1) {$\QQ$};
		\node[left=.4 of Q1] (One) {$\Sky_K\big(1$};
		\node[below right=-.3 and 0 of Prop1] (Prop2) {$\Bool\big)$};
		\draw[->] (Q1) -- (Prop2);
		\draw[->] (One) to node[above, font=\scriptsize] {$q$} (Q1);
	\end{scope}
	\begin{scope}[font=\normalsize]
		\node[below left=-.3 and 0 of One] (Q2) {$1$};
		\node[below right=-.3 and 0 of Prop2] (Bool) {$\Sky_K(\Bool)$};
		\node[right=2 of Bool] (Prop) {$\Prop$};
		\draw[->] (Q2) -- (Bool);
		\draw[->] (Bool) to node[above, font=\footnotesize] {$\classify{\Sky_{K}(\top)}$} (Prop);
	\end{scope}
\end{tikzpicture}
\]


\todo[inline, author=David]{Tobias, everything starting from the beginning of \cref{sec.main_thm} is new; what do you think?}


\begin{theorem}\label{conj.may8}
Let $\BaseSpace$ be a compact Hausdorff space and $X\in\Shv(C\BaseSpace)$. Then there is a bijection
\begin{multline*}
  \{\tn{internal probability valuations } \mu\colon \Prop^X \to\tLR\}
	\cong\\[4pt]
	\{\tn{external probability valuations } \mu'\colon 2^{X(B)} \to\LR\}
\end{multline*}
\end{theorem}

One special case is when $X$ has no global sections, so that the external set is empty. Another special case is \cref{prop.unique_probability_valuation_on_prop}, namely for $X = 1$ the terminal sheaf.

The proof crucially relies on \cref{thm.discrete_internal_space}. For an object $X$, the $\At{B}$-sheafification is the object of all predicates $P : X \to \Prop$ which are $\At{B}$-closed and $\At{B}$-local singleton. 

\begin{proof}
Starting with $\mu$ and given $S \subseteq X(B)$, the subsheaf of $X$ generated by $S$ is a global section of $\Prop^X$.
It is convenient to identify $\smset$ with the category of $\At{B}$-sheaves in $\cat{E}$, and the external reals $\LR$ with $\tLR_{\AtInline{B}}$. In other words, we identify external valuations with \emph{internal} maps
\[
	\mu' : \Prop_{\AtInline{B}}^{\asSh_{\AtInline{B}}(X)} \to \tLR_{\AtInline{B}}
\]
which are probability valuations in the subtopos of $\At{B}$-sheaves.

So suppose that we start with a given $\mu$. Then we construct $\mu'$ as follows. The $P : \asSh_{\AtInline{B}}(X) \to \Prop_{\AtInline{B}}$ are in (internal) bijection with the $P' : X \to \Prop_{\AtInline{B}}$ via composing with the sheafification map $\toSh : X \to \asSh_{\AtInline{B}}(X)$, or equivalently the $\At{B}$-closed predicates on $X$. Under this identification, we put $\mu'(P) \coloneqq \At{B} \mu(P')$.

\todo[inline]{Check that this has the required properties.

For $P':X\to\Prop_{\AtInline{B}}$, by \cref{prop.double_at} we have
\[\At{B}\mu(P')=\At{B}\mu(\At{}P')=\mu(\At{}P')=\mu(P').\]
The first three axioms follow, though I'm not sure about Scott continuity.
}

For the other direction, we start with $\mu'$ and construct a $\mu$ from it. Concretely, for any $q\in \tQQ$ and $P : X\to \Prop$, we consider the predicate $CB \to \Prop$ given by
\[
	K \longmapsto \exists U \ldotp \At{K} U \wedge \mu'(\Const(\In{U} P))(q),
\]
where $\mu'$ takes values in constant lower reals, which are a subobject of all lower reals.

and claim that it corresponds to a proposition via \cref{prop:prop_as_opens}. We thus need to check the openness condition of \cref{prop:prop_as_opens}. This means that we assume that the predicate holds on the given $K$, so that we have $U$ with $\At{K} U$ and $\mu'(\At{B} \In{U} P)(q)$. We then use this very $U$ in the openness condition. It then needs to be proven that $\At{K'} U$ implies that there is $U'$ with $\At{K'} U'$ and $\mu'(\At{B} \In{U'} P)(q)$; but then again we can take $U' \coloneqq U$, in which case there is nothing to prove. We thus conclude that the above predicate defines a proposition, and this is our $\mu(P)(q)$. Looking at the proof of \cref{prop:prop_as_opens}, we have concretely
\[
	\mu(P)(q) := \forall K \, \exists U \ldotp \At{K} U \wedge (\At{B} U \imp \mu'(\At{B} \In{U} P)(q)).
\]
\todo[inline]{Check that this $\mu$ has the required properties}

\todo[inline]{I'm switching back and forth between $\At{B}$ and $\See{B}$. Settle on one of them}

We now verify that these two constructions are inverse of one another. When starting with $\mu'$, we need to prove that for all $\At{B}$-closed predicates $P$,
\[
	\mu'(P)(q) = \forall K \, \exists U \ldotp \At{K} U \wedge (\At{B} U \imp \mu'(\At{B} \In{U} P)(q)).
\]
But since $P$ is $\At{B}$-closed, we have $\In{U} P = \In{U} \At{B} P = \At{B} P = P$. This reduces the problem to proving that
\[
	\mu'(P)(q) = \forall K \, \exists U \ldotp \At{K} U \wedge (\At{B} U \imp \mu'(P)(q)).
\]
This in turn is equivalent to
\[
	\At{K} \mu'(P)(q) = \exists U \ldotp \At{K} U \wedge (\At{B} U \imp \mu'(P)(q)),
\]
which follows from \cref{ax.at_to_nghbhd} upon using that $\mu'(P)(q)$ is $\At{B}$-closed.

Conversely, suppose that we start with $\mu$. Then we need to prove that for any $P : X \to \Prop$ and $q\in \tQQ$,
\[
	\mu(P)(q) = \forall K \, \exists U \ldotp \At{K} U \wedge (\At{B} U \imp \At{B} \mu(\At{B} \In{U} P)(q)),
\]
or equivalently that
\[
	\At{K} \mu(P)(q) = \exists U \ldotp \At{K} U \wedge (\At{B} U \imp \At{B} \mu(\At{B} \In{U} P)(q)),
\]
\todo[inline]{Problem. now we cannot apply \cref{thm.discrete_internal_space}}
\end{proof}

\begin{proposition}
\cref{conj.may8} holds in the case that $p\colon U\ss CB$ is the inclusion of an open set.
\end{proposition}
\begin{proof}
If $U=CB$ then there is exactly one section, so there is exactly one external valuation, and the result follows from \cref{prop.unique_probability_valuation_on_prop}.

So suppose that $U\neq CB$. Then there are no external probability valuations because the set $\Gamma(p)$ is empty. We suppose for contradiction that $\mu\colon p_*\Omega_U\to\tLR$ is an internal valuation. Then composing with the unique internal frame homomorphism $!\colon\Prop\to p_*\Omega_U$, we have an internal valuation $\mu\circ !$ on $\Prop$. By \cref{prop.unique_probability_valuation_on_prop}, we have $\mu\circ!=i$, where $i\colon\Prop\to\tLR$ is the indicator function.

Clearly $\church{i}(U)\neq \church{i}(CB)$, because these functions differ on points in the complement of $U$. But $\church{!}(U)=U\cap U=U\cap CB=\church{!}(CB)$, so $\church{\mu\circ !}(U)=\church{\mu\circ!}(CB)$, a contradiction.
\end{proof}




\section{Working externally}

Throughout this section, $B$ is locally compact, so $CB$ is a domain (and hence regularly based).

Consider the discrete frame $\Omega^X$ on a sheaf $X$ on $CB$. Given a point $K\in CB$ there is an adjunction
\begin{equation}\label{eqn.stalk}
\adj[50pt]{\Sub(X)}{S\mapsto S^K}{T\mapsto T_K}{2^{X_K}}
\end{equation}
where $S^K$ is the stalk of $S\in\Sub(X)$ at $K$. The subsheaf $T_K$ is given by
\[
  T_K(U)\coloneqq
  \begin{cases}
  	X(U)& \tn{ if }K\not\in U\\
		\{x\in X(U)\mid x_k\in T\}&\tn{ if }K\in U
  \end{cases}
\]
The induced monad on $\Sub(X)$ is $\At{K}$.

\subsection{Internal to external}
Given an internal valuation $\mu$ on $\Omega^X$, we get an external valuation $\mu_K$ on the set $2^{X_K}$, where $X_K$ is the stalk of $X$ at $K$. Indeed, for any subset $T\ss 2^{X_K}$, it is given by the formula
\[\mu_K(T)\coloneqq\mu(T_K)(K).\]

We prove it is indeed a valuation in the following proposition.
\begin{proposition}\label{prop.int_to_ext_measure_pres}
Let $\mu$ be an internal probability valuation on $X\in\Shv(CB)$. It induces a valuation $\mu_K$ on $2^{X_K}$ for every $K\in CB$, and whenever $K\ss K'$, the inverse image map $\rho_{K,K'}\inv\colon 2^{X_K}\to 2^{X_{K'}}$ is measure-preserving.
\end{proposition}
\begin{proof}
The maps $\mu_K$ and $\mu_{K'}$ can be written as composites, as shown in the following diagram:
\begin{equation}\label{eqn.measure_pres}
\begin{tikzcd}
	2^{X_K}\ar[d, "\rho_{K,K'}\inv"']\ar[r, "T\mapsto T_K"]\ar[dr, phantom, near start, "\Downarrow"]&
	\Sub(X)\ar[r, "\mu"]&
	\lsc(CB)\ar[r, bend left, "\ev_K"]\ar[r, bend right, "\ev_{K'}"']\ar[r, phantom, "\Uparrow"]&
	\RR\\
	2^{X_{K'}}\ar[ur, "T\mapsto T_{K'}"']&{}
\end{tikzcd}
\end{equation}
where $\lsc(CB)$ is the set of lower semicontinuous functions on $CB$. The lefthand 2-morphism says that for any $T\ss X_K$ we have $T_K\ss \rho_{K,K'}\inv(T)_{K'}$, which is obvious from the formula in both the case that $K'\not\in U$ and the case that $K'\in U$. The righthand 2-morphism is lower semicontinuity.

We first show that all four maps in \cref{eqn.measure_pres} are valuations. Each is a composite of several Scott continuous functions, each of which preserves the top element (more precisely, $\ev$ sends the top element to 1. We need to show that the composites send the bottom element to 0 and are modular. The first part follows from \cref{prop.unique_probability_valuation_on_prop} since $\mu(\varnothing_K)=\varnothing_K$. The second part follows from the fact that $T\mapsto T_K$ (and obviously $T\mapsto T_{K'}$) and $\rho_{K,K'}\inv$ preserve binary meets and joins.\todo{Check}

Finally, we prove that the map $2^{X_K}\to 2^{X_{K'}}$ is measure preserving. This amounts to showing that for any $T\ss X_K$ we have $\mu_K(T)=^?\mu_{K'}(\rho_{K,K'}\inv(T))$. By \cref{eqn.measure_pres} we have
\[\mu(T_K)(K)\leq\mu(\rho_{K,K'}\inv(T)_{K'})(K)\leq\mu(\rho_{K,K'}\inv(T)_{K'})(K')\]
but this also holds when $T_K$ is replaced by its complement $T_K^c$. The result then follows from inclusion-exclusion.

\end{proof}

\subsection{External to internal}
Given an external valuation $\mu'\colon 2^{X(B)}\to\RR$, define an internal valuation $\mu\colon\Sub(X)\to\tLR$ whose $U$-component for open $U\ss CB$ is given by the lower-real
\[\mu_U(S)(K)\coloneqq\mu'(\rho_{K,B}(S^K))\]
where $S\mapsto S^K$ is the stalk functor, as given in \cref{eqn.stalk}.


\subsection{Round trips}
We are most concerned with the round trip from internal to internal valuations. Starting with an internal valuation $\mu$, an open set $U\ss CB$, a subsheaf $S\ss X$, and a point $K\in CB$, the desired equation between $\mu$ and the internalization of its externalization is:
\[
\mu_U(S)(K)=^?\mu_B((\rho_{K,B}\inv(S^K))_B)(B)
\]
But by \cref{prop.int_to_ext_measure_pres}, we have $\mu_B((\rho_{K,B}\inv(S^K))_B)(B)=\mu_B((S^K)_K)(K)=\mu_B(\At{K}(S))(K)$. The result follows by Nelson's principle, $\mu((S^K)_K)(K')=\mu(S)(K')$ for any $K\ss K'$.



















\chapter{Correspondence with stochastic processes}

Here, we will investigate how probability valuation's on spaces of sections can be identified with stochastic processes in the conventional sense.

\chapter{Open questions}

Do probability valuations internal to some topos $\cat{E}$ give rise to a probability monad?



\newpage

\appendix


\appendix

\chapter{Unsorted material}

\section{Modalities preserve infima}

For any predicate $f:\Prop\to\Prop$, the infimum is $\forall(P:\Prop)\ldotp(fP\imp P)$.

\begin{lemma}
For any modality $j$, the type $\Prop_j$ is closed under infima. That is, for any $f:\Prop_j\to\Prop$
\[(j\forall(P:\Prop_j)\ldotp fP\imp P)\imp\forall(P:\Prop_j)\ldotp fP\imp P.\]
\end{lemma}
\begin{proof}
Assume $j\forall P\ldotp fP\imp P$, and move the $j$ inside the $\forall$ and the implication, to obtain $\forall P\ldotp fP\imp jP$. To prove $\forall(P:\Prop_j)\ldotp fP\imp P$, take any $P:\Prop_j$. Then we have $jP$ by assumption, which implies $P$ by $j$-closure.
\end{proof}

\section{Internalizing sups}
 
The (internalized) sup is just shorthand: $\sup_{U\ni K}X(U)\coloneqq \exists U\ldotp \At{K}U\wedge X(U)$.

One issue is ``internalizing the sup''. We think that works, but it's not technically proven. The argument is basically that we can check the following external fact holds about points contained in opens:
\[\big(K\in\church{\exists(U)\ldotp\At{K}U\wedge\In{U}P}\big)\iff\big(\text{there exists $U$ such that $K\in U$ and }K\in\church{\In{U}P}\big).\]

\section{Preservation of internal adjunctions}

All left exact functors $F$ preserve internal adjunctions $L\dashv R$. The finite limit preservation ensures that the $F$-image of a category object is a category object, and similarly for functors. But once you have that, the idea is to construct the unit and counit and see that they're preserved. Write down the diagram
\[
\begin{tikzcd}
	\Ob(C)\ar[drr, bend left, "RL"]\ar[ddr, bend right, equal]\ar[dr, "\eta"]\\
	&\Mor(C)\ar[r, "cod"]\ar[d, "dom"]&\Ob(C)\\
	&\Ob(C)
\end{tikzcd}
\]
and similarly for $\epsilon$, and assert the triangle identities; all of this structure will be preserved by $F$.


\section{Internal posets}

\begin{notation}
We write $\tQQp\coloneqq\{q:\tQQ\mid q\geq0\}$. Given a function $P:X\to\Prop$, we write $\subtype{P}\ss X$ to denote the associated subtype, $\subtype{P}\coloneqq\{x:X\mid Px\}$.
\end{notation}

\begin{definition}
Let $X$ be a type. A \emph{poset structure on $X$} is a function $L:X\times X\to\Prop$ satisfying
\begin{enumerate}
	\item $\forall(x:X)\ldotp L(x,x)$.
	\item $\forall(x,y,z:X)\ldotp L(x,y)\imp L(y,z)\imp L(x,z)$.
\end{enumerate}
We usually denote $L(x,y)$ as $x L y$. We call $(X,L)$ a \emph{poset}.
\end{definition}

In terms of sheaf semantics on some site $(\C,J)$, an internal poset is the same thing as a sheaf with values in $\Poset$; since the theory of posets is essentially algebraic, this is a consequence of Diaconescu's theorem, but it is also instructive to check it by hand.

\begin{example}
For any type $Y$, the type $\Phi\coloneqq (Y\to\Prop)$ has a natural poset structure given by $\lambda(f,g:Y\to\Prop)\ldotp\forall(y:Y)\ldotp (fy\imp gy)$.
\end{example}

\begin{example}
The type $\tLR$ of nonnegative lower real numbers is defined as the type of functions $\delta:\tQQp\to\Prop$ satisfying the following:
\begin{description}
	\item[\quad\parbox{1in}{Down-closed:}] $\forall(q,q':\tQQp)\ldotp (q<q')\imp\delta q'\imp\delta q$.
	\item[\quad\parbox{1in}{Rounded:}] $\forall(q:\tQQp)\ldotp\delta q\imp\exists(q':\tQQ)\ldotp(q<q')\wedge\delta q'$.
\end{description}
The type $\tLR$ has the structure of an ordered semi-ring. That is, it has the structure of
\begin{itemize}
	\item a poset: $\delta\leq \delta'\iff\forall q\ldotp\delta q\imp\delta'q$,
	\item a commutative monoid $(0,+)$:
	\begin{itemize}
		\item identity $0$ given by $\lambda q\ldotp q < 0$,
		\item  addition given by $(\delta_1+\delta_2)q\iff\exists(q_1,q_2)\ldotp(q<q_1+q_2)\wedge\delta_1q_1\wedge\delta_2q_2$,
	\end{itemize}
	\item a commutative monoid $(1,*)$:
	\begin{itemize}
		\item identity $1$ given by $\lambda q\ldotp q<1$,
		\item multiplication given by $(\delta_1*\delta_2)q\iff\exists(q_1,q_2)\ldotp(q<q_1*q_2)\wedge\delta_1q_1\wedge\delta_2q_2$,
	\end{itemize}
\end{itemize}
where multiplication distributes over addition, and both preserve order.
\end{example}

\begin{definition}
Let $(X,\leq)$ be a poset and let $P:X\to\Prop$ be a subtype. A \emph{join} of $P$ is an element $s_P:X$ satisfying
\begin{enumerate}
	\item $\forall(x:\subtype{P})\ldotp x\leq s_P$.
	\item $\forall(s':X)\ldotp[\forall(x:\subtype{P})\ldotp x\leq s']\imp s_P\leq s'$.
\end{enumerate}
Similarly, a \emph{meet} of $P$ is an element satisfying the dual statements (the arguments to every $\leq$ symbol are swapped).

We say that $X$ has \emph{binary joins} (resp.\ \emph{binary meets}) if, for every $x_1,x_2:X$ the subtype $\lambda(x:X)\ldotp (x=x_1) \vee (x=x_2)$ has a join, in which case we denote it $x_1\vee x_2$ (resp.\ meet $x_1\wedge x_2$). 

We say that $X$ has a \emph{bottom element} if $\lambda(x:X)\ldotp \bot$ has a join, in which case we denote it $\varnothing:X$. We say that $X$ has \emph{finite joins} if $X$ has binary joins and a bottom element (resp.\ binary meets and a top element). We say that $X$ has \emph{directed joins} if, every directed $P$ has a join, denoted $\sup P$.
\end{definition}

\begin{example}
For any type $Y$, the poset $\Phi\coloneqq Y\to\Prop$ has all joins and meets. The join of $P:\Phi\to\Prop$ is $s_P\coloneqq\lambda(y:Y)\ldotp\exists(\phi:\subtype{P})\ldotp\phi y$. It is easy to check that this satisfies the conditions of being a join. The meet of $P$ is $\lambda(y:Y)\ldotp\forall(\phi:\subtype{P})\ldotp\phi y$.
\end{example}

\begin{example}
The poset $(\tLR,\leq)$ has arbitrary joins: the join of $P:\tLR\to\Prop$ is $\lambda q\ldotp\exists(\delta:\subtype{P})\ldotp\delta q$. Note that $0$ is the bottom element. It also has binary meets: $\min(\delta_1,\delta_2)q\iff(\delta_1 q\wedge \delta_2q)$.

In fact, it also has arbitrary meets, though I don't think we'll need that. The meet of $P:\tLR\to\Prop$ is $(\bigwedge P)q\iff\exists q'\ldotp (q<q')\wedge\forall(\delta:\subtype{P})\ldotp\delta q'$.
\end{example}
\section{Posites}

\begin{proposition}
If $(S,\sqss)$ is a domain, then it has posite structure whose covering families are the way-up-closures.

Suppose that these families are directed, i.e.\ that if $s\ll s'_1$ and $s\ll s'_2$ then there exists $s\ll s'$ with $s'\sqss s'_1$ and $s'\sqss s'_2$. Then the Kripke-Joyal semantics for $\vee$ degenerates: $s\Vdash\phi\vee\psi$ iff $s\Vdash\phi$ or $s\Vdash\psi$.
\end{proposition}
\begin{proof}
The fact that $(S,\sqss)$ has the structure of a posite whose covering families are way-up closures is \cite[Remark 2.21]{Schultz.Spivak:2017a}.

The Kripke-Joyal semantics for $\vee$ is that $s\Vdash\phi\vee\psi$ iff there exists a covering family $\{s_i\to s\}_{i\in I}$ such that $s_i\Vdash\phi$ or $s_i\Vdash\psi$ for each $i\in I$. Here we have only one covering family, namely $\{s'\mid s\ll s'\}$ for each $s$. Suppose that $\upclose s$ is filtered and that $s\Vdash\phi\vee\psi$. Arguing classically, if $s\not\Vdash\phi$ and $s\not\Vdash\psi$ then there is some $s'_1,s'_2\in\upclose s$ such that $s'_1\not\Vdash\phi$ and $s'_2\not\Vdash\psi$. Let $s\ll s'\sqss \{s_1,s_2\}$. Then by monotonicity $s'\not\Vdash\phi$ and $s'\not\Vdash\psi$, a contradiction.
\end{proof}

\section{Locally compact spaces as continuous posets}
Something like the following should be true.

Suppose $X$ is a topological space and $\Op(X)$ is its frame of opens. A Scott-open in $\Op(X)$ is a subset $K\ss\Op(X)$ that is up-closed and inaccessible by directed joins: for any directed collection $I\ss\Op(X)$ of frame elements, if $(\bigvee_{i\in I} U_i)\in K$ then $U_i\in K$ for some $i\in I$. These are the filters corresponding to compact sets.

Given opens $U,V\in\Op(X)$, we have $U\ll V$ (i.e.\ $U$ is way-below $V$ in the frame) iff, at the level of the topological space there exists a compact subset $K\ss X$ with $U\ss K\ss V$.



\section{Connectedness}

\begin{definition}
	We say that $P : \Prop$ is \emph{connected} if
	\begin{align*}
		\conn{P} \: := \: \forall Q_1, Q_2 : \Prop \ldotp (P \imp Q_1 \lor Q_2) \Longrightarrow & (Q_1 \land Q_2 \imp \bot) \\
			& \Longrightarrow \left[(P \imp Q_1) \lor (P \imp Q_2) \right]
	\end{align*}
	holds.
\end{definition}

\tob{Not yet sure if this is ``right'', since it gives $\conn{\bot}$, in contrast to the usual wisdom which says that the empty space is not connected}

By extension, we also define connectedness of predicates: for $P : X \to \Prop$,
\[
	\conn{P} := \forall x : X\ldotp \conn{P(x)}.
\]

\section{Regularly based}
]
We say that a topological space is \emph{regularly-based} if it has a basis of regular opens $U$, i.e.~open sets which are the interior of their closure, or equivalently which are their own double pseudo-complement, $\neg\neg U=U$.

\begin{proposition}
If $\BaseSpace$ is any space, then $C\BaseSpace$ is regularly-based.
\end{proposition}

\begin{proof}
We first show that the closure $\ol{U_{C\BaseSpace}}$ of the basic open $U_{C\BaseSpace}$ consists of all compacts $K_{C\BaseSpace}$ with $K\cap\overline{U} \neq \emptyset$. Clearly the complement of this set $\{K_{C\BaseSpace}\mid K\cap\ol{U}=\emptyset\}$ is open, since it is the open $(\ol{U}^c)_{C\BaseSpace}$ associated to the complement of $\overline{U}$. Moreover, it is the largest open with this property: if $V\subseteq B$ is any open for which $K_{C\BaseSpace}\in V_{C\BaseSpace}$ implies $K_{C\BaseSpace} \not\in U_{C\BaseSpace}$, then $V$ must be disjoint from $U$, since any single point $\{p\}\ss(V\cap U)$ is itself a nonempty compact set. This proves the first claim.

Next, we show that the interior of the above closure is $\{ K \mid K\subseteq U\}$ again. We need to show that it is not larger; so for a given $K$ in this interior, we need to prove $K\subseteq U$. Being in the interior of the above closure means that there is an open $V$ with $K\subseteq V$ such that every $K' \subseteq V$ is in the closure, which means $K' \cap \overline{U} \neq \emptyset$. Again since points are compact, we get $V \subseteq U$, and therefore $K \subseteq U$, as was to be shown.

Since forming the interior of the closure is an idempotent operation (corresponding to double negation), the claim follows.
\end{proof}
Equivalently, we could have said that every $P:\Prop$ must be the supremum of all $Q:\Prop$ with $Q\imp P$ and $\neg\neg Q\imp Q$. For topological spaces, recall that an open set $U$ in a topological space is called \emph{regular} if it is the interior of its closure, or in other words $\neg\neg U=U$. Because $\neg\neg$ is a dense modality, the set of regular opens contains the whole space, the empty set, and is closed under finite intersections. So for the topos of sheaves on a space, our condition expresses the requirement that the regular opens form a basis, i.e.\ for all $P\in\Op$, we have $P=\bigcup {\{Q\ss P\mid \neg\neg Q=Q\}}$.

\begin{remark}
In the book \emph{Counterexamples in Topology} they define semi-regular spaces to be Hausdorff spaces that are regularly-based in our sense above. We are certainly interested in cases where the space is not Hausdorff, e.g.\ $\IR$.

It is not hard to show however that any regular space is regularly-based in our sense. 
\end{remark}


\section{The valuations profunctor}

If $\cat{C}$ is a category with finite limits and small colimits then one can define frames and symmetric monoidal DCPOs internal to $\cat{C}$. From here, one can define normalized complete valuations (probability valuations) as a profunctor
\[V_{\cat{E}}\colon\Fun{Frm}(\cat{E})\tickar\Fun{SMonDCPO}(\cat{E})\]
In other words, one can precompose a probability valuation with a frame morphism on the left or post-compose with a monoidal DCPO map on the right. We will usually refer to this profunctor more simply as the \emph{valuations profunctor}.

The question is how direct image functors $f_*\colon\cat{E}\to\cat{E}'$ interact with the valuations profunctor. What we want is a 2-cell in the double category of categories, functors, and profunctors:
\begin{equation}\label{eqn.direct_image_valuations}
\begin{tikzcd}
	\Fun{Frm}(\cat{E})\ar[r, tick, "V_{\cat{E}}"]\ar[d, "\Fun{Frm}(f_*)"']&\Fun{SMonDCPO}(\cat{E})\ar[d, "\Fun{SMonDCPO}(f_*)"]\\
	\Fun{Frm}(\cat{E}')\ar[r, tick, "V_{\cat{E}'}"']&\Fun{SMonDCPO}(\cat{E}')\ar[ul, phantom, "?\Downarrow"]
\end{tikzcd}
\end{equation}
where the functors on the left and right were (hopefully) established in \cref{sec.preserve_loc_frame}. Intuitively, given a valuation $\mu\colon F\to R$ upstairs, we want to define a valuation $\mu'\colon f_*F\to f_*R$ downstairs. There is an obvious candidate, namely $\mu'\coloneqq f_*\mu$. The question is whether this preserves the various structures.

It should be obvious that $\mu'$ is monotonic. We believe that $\mu'$ will commute with directed joins by the work in \cref{sec.preserve_loc_frame}. Both the top and bottom element of an internal frame can be characterized by an equational theory, so they should be preserved by left-exact functors such as $f_*$. It remains to show that $\mu'$ satisfies inclusion-exclusion.

To begin with, we need to know that $f_*$ commutes with binary joins; but what does this mean? We want to define binary joins equationally on a frame $F$ as $\vee\colon F\times F\to F$, and then ask that $f_*(\vee_F)=\vee_{f_*F}$, but then the question becomes ``in what sense is the binary join operation the same as the actual binary join?'' In other words, what if we're only preserving a structure, not the universal property it is supposed to satisfy?

For any pair of objects $A^B$ in a topos there is a map $\im\colon A^B\to\Prop^A$, that ``sends a map to its image''. Namely, use the predicate $I\colon A\times A^B\to\Prop$ given by $I(a,f)=\exists(b:B)\ldotp f(b)=a$. Thus we get a diagram, and the above questions are whether it commutes:
\[
\begin{tikzcd}
	F\times F\ar[d, equal]\ar[r, "\vee"]&F\\
	F^{1+1}\ar[r, "I"']&\Prop^F\ar[u, "\bigvee"']\ar[ul, phantom, "?"]
\end{tikzcd}
\]
The goal then is to encode the universal property itself in an equational way. Take the graph of the function $\vee\colon F\times F\to F$ and call it $G(\vee)\ss F\times F\times F$. There are two equations that characterize it as the graph of a function, so we know that $f_*$ will send it to the graph of a function, which must again be $f_*(\vee_F)$. We want to characterize its universal property. We thought it is roughly the following:
\[
\begin{tikzcd}
	\{w,x,y,z:F\mid x\leq z, y\leq z, w=x\vee y\}\ar[r]\ar[d]\ar[dd, shift right=40pt, bend right=60pt, dotted]
	&\{w,x,y,z:F\mid (x,y,w)\in G(\vee)\}\ar[d]\\
	\{w,x,y,z:F\mid x\leq z, y\leq z\}\ar[r]&\{w,x,y,z:F\}\ar[ul, phantom, very near end, "\lrcorner"]\ar[d]\\
	\{w,z\mid w\leq z\}\ar[r]&\{w,z:F\}
\end{tikzcd}
\]

Now we have that the direct image map $f_*$ preserves everything that's equationally defined, so we are willing to bet on the following.
\begin{proposition}
A direct image functor $f_*\colon\cat{E}\to\cat{E'}$ preserves valuations in the sense that it induces a 2-cell as shown in \cref{eqn.direct_image_valuations}.
\end{proposition}
\chapter{Toy example}

\todo[inline]{We can recycle some of the following to use as our running example for illustration}

How about considering only two instances of time first, so that we work in the presheaf topos over the discrete two-point space $\{a,b\}$? Here, we will only need three modalities, where $\pi$ corresponds to sheafification, and $\At{a}{}$ and $\At{b}{}$ correspond to behavior at $a$ and $b$, respectively. In this case, measures on types (or suitable locales) which satisfy $\pi$ should semantically correspond to pairs of random variables equipped with a joint distribution. This is a good test case where the essential structures already come up, but without the technical complexity of TTT.

This is the topos $\Span\set$, where $\Span$ is the category \fbox{$a\from ab\to b$}. The subobject classifier for $\Span\set$ has five global sections, which we can represent as follows:
\[\church{\Prop}(ab)=\{\bot,a,b,a\vee b,\top\}.\]
We have $a=(\neg b)=(b\imp a)$ and similarly $b=(\neg a)=(a\imp b)$.

Here is a type theory for $\Span\set$: no atomic types, no atomic terms, two atomic propositions $a,b:\Prop$, and two axioms: $a=\neg b$ and $b=\neg a$. Note that $a=\neg\neg a$ and similarly $b=\neg\neg b$, but not so for $a\vee b$; the logic is not boolean.

The analogous modalities to those used in TTT are:
\[
\begin{array}{c|c|rccc}
	\textbf{modality}&\textbf{definition on } P:\Prop&\textbf{values: }\bot&a&b&a\vee b \\\hline
	\In{a}&a\imp P&b&\top&b&\top\\
	\See{a}&b\vee P&b&a\vee b&b&a\vee b\\
	\At{a}&(P\imp b)\imp b&b&\top&b&\top\\\hline
	\In{b}&b\imp P&a&a&\top&\top\\
	\See{b}&a\vee P&a&a&a\vee b&a\vee b\\
	\At{b}&(P\imp a)\imp a&a&a&\top&\top\\\hline
	\pi&\At{a}P\wedge\At{b}P&\bot&a&b&\top
\end{array}
\]

\begin{remark}
Semantically we see that $\church{\In{a}}=\church{\At{a}}$, but in fact, our axioms may not be strong enough to prove that internally. It is easy to show $\forall(P:\Prop)\ldotp(a\imp P)\imp(P\imp b)\imp b$, but I do not see how to show $\forall(P:\Prop)\ldotp[(P\imp b)\imp b]\imp^? (a\imp P)$. Without additional axioms, we have to decide whether to work with $\At{a}$ or $\In{a}$ in our notion of measure; we use $\At{}$ so as to remain faithful to the TTT case. 
\end{remark}

Although defined as the topos on presheaves on the discrete two-point space $\{a,b\}$, our topos $\Span\set$ can also be viewed as the topos of \emph{sheaves} on a topological space $W$, with three points $a,b,ab$ and four open sets: $\emptyset$, $W$, $\{a\}$ and $\{b\}$. The type $\tRR$ corresponds to the constant sheaf, by general theory of real number objects in localic toposes.

\begin{theorem}
Given finite sets $A$ and $B$, valuations on the power object of the $\pi$-type given by the product projections span
\[\begin{tikzcd}
	& A \times B \ar{dl} \ar{dr} \\
	A & & B
\end{tikzcd}\]
are semantically in bijective correspondence with probability measures on $A\times B$.
\end{theorem}

Here, our valuations take values in the lower reals.

We also write $(A,B)\in\Span\set$ as shorthand for the above span. So the global elements of the object of valuations on $(A,B)$ are precisely the probability measures on $A\times B$.

\begin{proof}
The lower reals are given by the span
\[\begin{tikzcd}
	& \{ (x, y, z) \in\mathbb{R}^3 \:\mid\: x,y\geq z \} \ar[swap]{dl}{(x,y,z)\mapsto x} \ar{dr}{(x,y,z)\mapsto y} \\
	\mathbb{R} & & \mathbb{R}
\end{tikzcd}\]
The power object $\Prop^{(A,B)}$ is given by the span
\[\begin{tikzcd}
	& Nat(A,B) \ar{dl} \ar{dr} \\
	\{\bot,\top\}^A & & \{\bot,\top\}^B
\end{tikzcd}\]
where $Nat(A,B)\subseteq \{\bot,a,b,a\lor b,\top\}^{A\times B}$ is the subset consisting of those functions $\phi : A\times B \to \{\bot,a,b,a\lor b,\top\}$ with the property that $\At{a} \phi(x,y)$ is independent of $y$, and likewise $\At{b} \phi(x,y)$ is independent of $x$. Hence the elements of $Nat(A,B)$ can be identified with triples $(S,T,U)$ where $S\subseteq A$ and $T\subseteq B$ as well as $U\subseteq S\times T$; the truth value of $(x,y)$ is $\geq a$ iff $x\in S$; it is $\geq b$ iff $y\in T$; and finally it is $\top$ iff $(x,y)\in U$. In this picture, the maps on the two legs are simply $(S,T,U)\mapsto S$ and $(S,T,U)\mapsto T$. The connectives are given componentwise,
\begin{align*}
	(S_1,T_1,U_1) \land (S_2,T_2,U_2) & = (S_1\cap S_2,T_1\cap T_2,U_1\cap U_2),\\
	(S_1,T_1,U_1) \lor (S_2,T_2,U_2)  & = (S_1\cup S_2,T_1\cup T_2,U_1\cup U_2).
\end{align*}
By all this, a global element of the object of normalized valuations is a diagram
\[\begin{tikzcd}
	& \{(S,T,U) \:\mid\: S \subseteq A , \: T \subseteq B , \: U \subseteq S\times T \} \ar{dl} \ar{dr} \ar{dd}{\mu_{ab}} \\
	\{ S \subseteq A \} \ar{dd}{\mu_a} & & \{ T \subseteq B \} \ar{dd}{\mu_b} \\
	& \{ (x,y,z) \in \mathbb{R}^3 \:\mid\: x,y\geq z \} \ar{dl} \ar{dr} \\
	\mathbb{R} & & \mathbb{R}
\end{tikzcd}\]
satisfying in addition monotonicity and the inclusion-exclusion and normalization equations. The commutativity of the diagram expresses the requirement that the first two components of $\mu_{ab}(S,T,U)$ must be equal to $\mu_a(S)$ and $\mu_b(T)$, respectively; in particular, if we replace $\mu_{ab}$ by its third component, we have the inequalities
\[
	\mu_{ab}(S,T,U) \leq \mu_a(S), \qquad \mu_{ab}(S,T,U) \leq \mu_b(T).
\]
So from now on, we will write $\mu_{ab}$ for only the third component, which is an external real number. By monotonicity of the valuation, it is enough to postulate these inequalities only in the special case $T = B$ and $U = S\times T$,
\begin{equation}
\label{marginaldomination}
	\mu_{ab}(S,B,S\times B) \leq \mu_a(S), \qquad \mu_{ab}(A,T,A\times T) \leq \mu_b(T),
\end{equation}
as long as each of $\mu_a$, $\mu_b$ and $\mu_{ab}$ is monotone (which is the case). Furthermore, we have normalization, which states $\mu_a(A) = \mu_b(B) = \mu_{ab}(A,B,A\times B) = 1$, and the inclusion-exclusion relation, which states the obvious for $\mu_a$ and $\mu_b$, as well as
\begin{equation}
\label{abinclexcl}
	\mu_{ab}(S_1\cap S_2,T_1\cap T_2,U_1\cap U_2) + \mu_{ab}(S_1\cup S_2,T_1\cup T_2,U_1\cup U_2) = \mu_{ab}(S_1,T_1,U_1) + \mu_{ab}(S_2,T_2,U_2).
\end{equation}
Moreover, we claim that together with normalization, this forces the inequalities~\eqref{marginaldomination} to be equalities. Indeed, we have as an instance of inclusion-exclusion,
\[
	\mu_{ab}(A\setminus S,B,(A\setminus S)\times B) + \mu_{ab}(S,B,S\times B) = \mu_{ab}(\emptyset,B,\emptyset) + \mu_{ab}(A,B,A\times B).
\]
The first term on the right-hand side vanishes since $0 \leq \mu_{ab}(\emptyset,B,\emptyset) \leq \mu_a(\emptyset) = 0$, while the second term is $1$ by normalization. With $A\setminus S$ in place of $S$, the first inequality~\eqref{marginaldomination} therefore gives
\[
	1 - \mu_{ab}(S,B,S\times B) \leq \mu_a(A\setminus S) = 1 - \mu_a(S).
\]
\todo[inline]{These seem to be correspond to the point-determination axiom, which therefore follows from normalization. Is this correct?}
Combining this with~\eqref{marginaldomination}, we conclude that the first inequality in~\eqref{marginaldomination} must hold with equality; the same reasoning applies to the second inequality. Therefore $\mu_a$ is already determined by $\mu_{ab}$, and so is $\mu_b$.

So overall, a normalized valuation on $(A,B)$ is the same thing as a map $(S,T,U)\longmapsto \mu_{ab}(S,T,U)$ with values in classical $[0,1]$ that satisfies monotonicity, inclusion-exclusion in the form~\eqref{abinclexcl}, and normalization $\mu_{ab}(A,B,A\times B) = 1$.

What is missing is to show that $\mu_{ab}(S,T,U) = \mu_{ab}(A,B,U)$ for all $S,T,U$, so that $\mu_{ab}$ depends only on $U\subseteq A\times B$, making it into a probability measure in the standard sense. Again by inclusion-exclusion and monotonicity, this equation is equivalent to $\mu_{ab}(A,B,\emptyset) = 0$. But this equation is also automatic thanks to inclusion-exclusion,
\[
	\mu_{ab}(A\cap \emptyset, \emptyset \cap B, \emptyset) + \mu_{ab}(A\cup \emptyset, \emptyset \cup B, \emptyset) = \mu_{ab}(A, \emptyset, \emptyset) + \mu_{ab}(\emptyset, B, \emptyset).
\]
We have already shown that both terms on the right-hand side must vanish, and the first term on the left vanishes trivially. Therefore $\mu_{ab}(A,B,\emptyset) = 0$, as was to be shown.
\end{proof}

What David sees as the most interesting part of the above proof is the last step, that $\mu_{ab}(S,T,U) = \mu_{ab}(A,B,U)$ for all $S,T,U$. Here's how that looks logically.

Recall that $\See{[0,n]}P\coloneqq(t\apart [n,0])\vee P$, where $t\apart[a,b]$ means $(b+1\leq t)\vee(t\leq a-1)$. So $\See{[0,n]}P=(1\leq t)\vee(t\leq n-1)\vee P$. Recall also that $\mu(P)$ is formally a function on rationals, $\mu(P):\tQQ\to\Prop$; if $\mu(P)(q)$ holds, we think ``$q<\mu(P)$''. So for example, down-closure says $(q'<q\wedge\mu(P)(q))\imp\mu(P)(q')$. In $\tRR_{\SeeInline{[0,n]}}$, the $\See{[d,u]}$-(nonnegative lower reals), $\See{[d,u]}\bot$ is the unit for addition.

\begin{theorem}\label{thm.logical_reformulation}
\[\See{[0,n]}\mu(P)=\See{[0,n]}\mu(\See{[0,n]}P).\]
\end{theorem}
\begin{proof}
We have $\mu(P)+\mu(\See{[0,n]}\bot)=\mu((\See{[0,n]}\bot)\wedge P)+\mu(\See{[0,n]} P)$. Since $\See{[0,n]}$ commutes with addition, it suffices to show $\See{[0,n]}\mu(\See{[0,n]}\bot)=\See{[0,n]}\mu((\See{[0,n]}\bot)\wedge P)$. But $\See{[0,n]}\bot\leq\See{[0,n]}\mu(\See{[0,n]}\bot\wedge P)\leq\See{[0,n]}\mu(\See{[0,n]}\bot)$, so it suffices to show that $\See{[0,n]}\mu(\See{[0,n]}\bot)\leq\See{[0,n]}\bot$. We also have
\[\mu(\See{[0,n]}\bot)+\mu(1\leq t\leq n-1)=\mu(1\leq t)+\mu(t\leq n-1),\]
so it suffices to show that $\See{[0,n]}\mu(1\leq t)\leq\See{[0,n]}\bot$ and $\See{[0,n]}\mu(t\leq n-1)\leq\See{[0,n]}\bot$. 

These are similar, so we show the first, and since $\See{[0,n]}$ is a modality, it suffices to show that $\mu(1\leq t)(q)\imp\See{[0,n]}\bot$ for any $q\in\QQ_{\geq0}$. So suppose $\mu(1\leq t)(q)$.

By an axiom of discrete temporal type theory (analogous to Axiom 3a in TTT), we have $\neg(t\leq 0)\imp (1\leq t)$ [and similarly $\neg(n\leq t)\imp(t\leq n-1)$]. Since $(1\leq t)\imp\See{[0,n]}\bot$, it suffices to show $\neg(t\leq 0)$, so assume $t\leq 0$. Then $\mu(1\leq t)=\mu(\bot)$, so $\mu(1\leq t)(q)=\mu(\bot)(q)=\bot$. This completes the proof.\end{proof}

\todo[inline]{Do we generally expect it to be the case that $\mu(P) = P$, or at least $\mu(P) \leq P$, for every proposition $P$? Because then we would get
\[
	\mu(1 \leq t) \leq \mu(\See{[0,n]} \bot) \leq \See{[0,n]} \bot,
\]
and therefore directly $\See{[0,n]} \mu(1 \leq t) \leq \See{[0,n]} \bot$, since applying $\See{[0,n]}$ to a lower real is the internal left adjoint to $\RR \rightarrow \RR_{\SeeInline{[0,n]}}$. (Right?) Or is this argument begging the question?}

\chapter{Domain-theoretic aspects of internal locales}

For now, this stuff is taking place in $\smset$.

Let $X$ be a topological space with poset of opens $\Op(X)$. Then for $U\subseteq V$ in $\Op(X)$, what does it mean for $U$ to be way below $V$? There is a nice way to relate it to the finite subcover definition of compactness:

\begin{lemma}
$U\ll V$ if and only if every open covering of $V$ has a finite subfamily which covers $U$.
\end{lemma}

\begin{proof}
If the condition holds, then every directed set of opens that goes below $V$ also covers $V$, and therefore contains an element that already contains $U$.

Conversely, given a covering family, consider the directed set of finite unions of members of the cover.
\end{proof}

Without additional separation assumptions on $X$, it seems tricky to translate this into a condition involving only the usual concepts of point-set topology. In order to do so, we need one more auxiliary statement.

\newcommand{\cl}[1]{\overline{#1}} % topological closure
\newcommand{\intr}[1]{\mathrm{int}(#1)} % interior

\begin{lemma}
Let $X$ be regular and $D\subseteq X$ dense. Suppose that every open cover of $X$ has a finite subfamily which covers $D$. Then $X$ is compact.
\end{lemma}

This may be somewhat surprising, since in many cases a finite subfamily which covers $D$ does not yet cover $X$.

\begin{proof}(Due to Ramiro de las Vegas, MO.)
Let $\mathcal{A}$ be an open cover of $X$. For every $x\in X$, choose $U_x\ni x$ such that $\cl{U_x}\subseteq V_x$ for suitable $V_x\in\mathcal{A}$. Then $\{U_x\}$ is another open covering which has a finite subcover $\{U_{x_1},\ldots,U_{x_n}\}$ by assumption. Since $X = \cl{U_{x_1}\cup\ldots\cup U_{x_n}} = \cl{U_{x_1}}\cup\ldots\cup\cl{U_{x_n}} \subseteq V_{x_1}\cup\ldots\cup V_{x_n}$, we are done.
\end{proof}

Now we can characterize the way below relation in a nice way:

\begin{lemma}
Let $X$ be regular. Then $U\ll V$ if and only if $U$ is relatively compact in $V$, i.e.~if $\cl{U}\cap V$ is compact.
\end{lemma}

\begin{proof}
If $\cl{U}\cap V$ is compact, then clearly the condition of the first lemma holds as well, and we do not need regularity.

Conversely, suppose that the condition of the first lemma holds. Then by the previous lemma, it is enough to show that every open cover of $\cl{U}\cap V$ has a finite subfamily which covers $U$. But this is clear by assumption: throwing in $V\setminus\cl{U}$ to an open cover of $\cl{U}\cap V$ gives an open cover of $V$.
\end{proof}

\chapter{Locales and geometric morphisms}

My question is: given a geometric morphism $f : \mathcal{T}\to\mathcal{T}'$, do internal frames in $\mathcal{T}'$ pull back to internal frames in $\mathcal{T}$?

Note that there is a natural comparison map between $f^*\Prop'$ and $\Prop$. Indeed, since $f^*$ preserves monos by virtue of being left exact, there is a map $\classify{f^*\const{true}'}\colon f^*\Prop'\to\Prop$ classifying the image under $f^*$ of $\const{true}'\colon 1'\to \Prop'$.

\begin{lemma}
If $s:A\to B$ is an arbitrary subobject in $\mathcal{T}'$, then $f^*s : f^*A\to f^*B$ is again a subobject, and $\classify{f^*s}$ is given by $f^*\classify{s}$ post-composed with the comparison map $f^*\Prop'\to\Prop$.
\end{lemma}

\begin{proof}
 This is because
\[\begin{tikzcd}
	f^*A \ar{r} \ar{d} & f^*B \ar{d} \\
	f^*1 \ar{r} & f^*\Prop'
\end{tikzcd}\]
is again a pullback diagram, and likewise is
\[\begin{tikzcd}
	f^*1 \ar{r} \ar{d} & f^*\Prop' \ar{d} \\
	1 \ar{r} & \Prop
\end{tikzcd}\]
so that the two pullbacks compose.
\end{proof}

In this reasoning, we have only needed to use that $f^*$ is left exact. Thus everything applies likewise to $f_*$, and we e.g.~have a comparison map $\classify{f_*\const{true}}\colon f_*\Prop\to\Prop'$. But this is of less interest to us here.

So given an internal poset $P$ in $\mathcal{T}'$, we get a poset structure on $f^*P$ given by the subobject $f^*(\leq) \to f^*P\times f^*P$. Again using the fact that $f^*$ preserves finite limits, it's straightforward to see that if $P$ is a meet-semilattice in $\mathcal{T}'$, then $f^*P$ is a meet-semilattice in $\mathcal{T}$. However, it's less clear if the existence of arbitrary joins is preserved in the same way.

The existence of arbitrary joins for $P = \mathcal{O}(X)$ means that the down-closure map
\[
	\mathcal{O}(X)\to\Prop'^{\mathcal{O}(X)},\qquad U\mapsto \classify{ \{ V \leq U \} }
\]
has an internal left adjoint. Here, $\Prop'^{\mathcal{O}(X)}$ carries the usual pointwise ordering, i.e.~the one corresponding to inclusion of subobjects. What we would like to have is that the down-closure map of $f^*\mathcal{O}(X)$ similarly has a left adjoint. According to the Elephant (p.~580), this is not the case in general! However for an \emph{atomic} geometric morphism---or in other words if $f^*$ is a logical functor---we can apply Corollary B2.3.10 to conclude that $f^*\mathcal{O}(X)$ is indeed an internal frame. Concretely, in this case the sup map is the composite $\Prop^{f^*(\mathcal{O}(X))}\cong f^*\left(\Prop'^{\mathcal{O}(X)}\right)\to f^*(\mathcal{O}(X))$, where the first morphism comes from $f^*$ being logical and the second one is $f^*(\sup)$.

In conclusion, internal frames can be pulled back along atomic geometric morphisms\footnote{And also along general geometric morphisms, but this requires a different construction, as in Elephant p.~580.}. Again by Elephant p.~580, pushing frames forward along $f_*$ is even simpler, and it seems plausible that this results in an adjunction $f^* \dashv f_*$ between frames in $\mathcal{T}$ and frames in $\mathcal{T}'$. I will be a bit sloppy for now and just assume that this is true.

The next question is whether $f^*$ preserves properties of locales such as regularity or local compactness. Both are concerned with relaxed notions of ordering:

\begin{definition}
For $U,V\in\mathcal{O}(X)$, one says that
\begin{enumerate}
\item $V$ is \emph{well inside} $U$, written $V\subset\subset U$, if there is $G\in\mathcal{O}(X)$ such that $V\cap G = \emptyset$ and $U\cup G = X$.
\item $V$ is \emph{way below} $U$, written $U\ll V$, if for every directed $D:\mathcal{O}(X)\to\Prop$, $U\leq\sup D$ implies that there is $W\in D$ with $V\leq W$.
\end{enumerate}
Then $X$ is regular (locally compact) if every $U$ is the sup of the $V$ that are well inside (way below) it.
\end{definition}

Since these definitions make sense constructively, we can apply them to internal locales.

So the well-inside relation corresponds to the subobject of $\mathcal{O}(X)\times\mathcal{O}(X)$ that is the projection of the subobject
\[
	\{ \: (V,G,U) \:|\: V\wedge G = \emptyset, \; U\vee G = X \: \} \quad \subseteq \quad \mathcal{O}(X)\times\mathcal{O}(X)\times\mathcal{O}(X)
\]
to the first and third factor, and this subobject itself is an equalizer. Since $f^*$ preserves monos and epis, it also preserves image factorizations and therefore pushforwards of subobjects, such as this pushforward along the product projection. Hence the well-inside relation on $f^*(\mathcal{O}(X))$ coincides with $f^*(\subset\subset)$.

Regularity means that upon exponentiating the first factor in $\classify{\subset\subset} : \mathcal{O}(X)\times\mathcal{O}(X)\to\Prop'$ to $\mathcal{O}(X)\to\Prop'^{\mathcal{O}(X)}$ and composing with $\sup$, one obtains $1_{\mathcal{O}(X)}$. Again since $f^*$ is logical and $f^*(\sup)$ is the sup map of $f^*(\mathcal{O}(X))$, we can conclude that $f^*(\mathcal{O}(X))$ is regular is well.

In order to make the same argument for local compactness, we show similarly that $f^*(\ll)$ is the way-below relation on $f^*(\mathcal{O}(X))$. By similar reasoning as for well-inside, we obtain that $\const{Drct}(f^*(\mathcal{O}(X))) = f^*(\const{Drct}(\mathcal{O}(X)))$.

\section{$f_*$ preserves internal locales and frames?}\label{sec.preserve_loc_frame}

\todo[inline]{Probably more correct discussion from the meeting:}

We want to understand why $f_*$ preserves internal locales---the proof in Johnstone is opaque---and similarly for internal frames. An internal poset $A$ has all sups if it is equipped with an adjunction
\[
\begin{tikzcd}
	A\ar[r, shift left, "\down"]&\Prop^A\ar[l, shift left, "\bigvee"]
\end{tikzcd}
\]
So suppose $A$ is an internal locale in $\cat{E}$ and that we have a geometric morphism $f_*\colon\cat{E}\to\cat{F}$. Applying our work above, we get an adjunction
\[
\begin{tikzcd}
	f_*A\ar[r, shift left, "f_*\down"]&f_*(\Prop^A)\ar[l, shift left, "f_*\bigvee"]
\end{tikzcd}
\]
in $\cat{F}$. We also have a map $e\colon f_*(\Prop^A)\to\Prop^{f_*A}$ given by $f_*$ of evaluation followed by the characteristic map for $f_*\true\colon 1\to f_*\Prop$. Thus it suffices to find a left adjoint for $e$.

We produced a candidate left adjoint $e'$ for $e$, but we did not prove it works. However, it seems to work when $f_*$ is a geometric inclusion, i.e.\ when $\cat{E}$ is the category of sheaves for some modality on $\cat{F}$. In that case, we think of $U\coloneqq f_*$ as the underlying presheaf map and $\asSh\coloneqq f^*$ as sheafification. The desired map $e'\colon\Sub_p(UA)\to U\Sub_s(A)$ sends a subpresheaf $X\ss UA$ to the sheaf-theoretic image of its adjunct $\asSh(X)\surj\bullet\inj A$. In other words, $e'$ takes a sub-presheaf to its sheaf-theoretic image; this construction should be left adjoint to $e$.

For an arbitrary $f$, we attempt to repeat this idea. To obtain a map $\Prop^{f_*A}\to f_*(\Prop^A)$, we need a map $A\times f^*(\Prop^{f_*A})\to\Prop$, i.e.\ a subobject of the domain. Form the following diagram, and take the subobject shown on the bottom row:
\[
\begin{tikzcd}
	&\bullet\ar[r]\ar[d, tail]\ar[dr, phantom, very near start, "\lrcorner"]\ar[dddl, two heads, bend right]&1\ar[d]\\
	&f^*(f_*A\times\Prop^{f_*A})\ar[r, "f_*(ev)"]\ar[d, "\cong"']&f^*\Prop\\
	&f^*f_*A\times f^*\Prop^{f_*A}\ar[d, "\epsilon"']\\
	\bullet\ar[r, tail]&A\times f^*\Prop^{f_*A}
\end{tikzcd}
\]
Note that in the subtopos case above, $f^*f_*A\to A$ is an iso, so there is no need for the epi-mono factorization.

We believe that direct image functors $f_*$ preserve the internal structures of DCPOs, complete sup-lattices, and frames. For example, rather than consider $\Prop^P$ we consider $\Idl(P)$ and try to lift the previously (semi-)defined adjunction, as shown:
\[
\begin{tikzcd}
	f_*(\Idl P)\ar[r, shift left, dotted]\ar[d, tail]&\Idl(f_*P)\ar[l, shift left, dotted]\ar[d, tail]\\
	f_*(\Prop^P)\ar[r, shift left]&\Prop^{f_*P}\ar[l, shift left]
\end{tikzcd}
\]


\chapter{Internal topological spaces}

\begin{definition}
\label{internal_space}
Let $X$ be a type. A \emph{topology} on $X$ is a function $\Op:(X\to\Prop)\to\Prop$ satisfying the following conditions:
\begin{enumerate}
	\item $\Op(\lambda(x:X)\ldotp\top)$.
	\item $\forall(\phi_1,\phi_2:X\to\Prop)\ldotp(\Op(\phi_1)\wedge\Op(\phi_2))\imp\Op(\lambda(x:X)\ldotp\phi_1 x\wedge\phi_2 x)$.
	\item Suppose given $I: \Op(X) \to \Prop$. Then $\Op(\lambda(x:X)\ldotp\exists(\phi:X\to\Prop)\ldotp I(\phi)\wedge\phi (x))$.
\end{enumerate}
In other words, the whole space $X$ is open, the intersection of two opens is open, and the union of any $I$-indexed set of opens is open, for any open system $I$. % It will be useful to define open systems: $\mathrm{OpSys}\coloneqq\{I:(X\to\Prop)\to\Prop\mid\forall \phi\ldotp I(\phi)\imp\Op(\phi)\}.$

We refer to a pair $(X,\Op(X))$, where $X$ is a type and $\Op(X)$ is a topology on $X$, as an \emph{internal topological space}.
\end{definition}

\begin{example}[Discrete topology]
For any type $X$, the \emph{discrete topology} on $X$ is given by $\Op(\phi)=\top$.
\end{example}

\begin{example}[Topology on $\tRR_j$]\label{ex.usual_topology_R}
Let $j$ be a modality and define $\Op:(\tRR_j\to\Prop)\to\Prop$ by
\[\Op(\phi)\coloneqq\forall(r:\tRR_j)\ldotp\phi r\imp\exists(\epsilon:\tRR_j)\ldotp\epsilon>0\wedge\forall(r':\tRR_j)\ldotp(-\epsilon<r-r'<\epsilon)\imp\phi(r').\]
It is easy to check condition 1. For condition 2, we obtain $\epsilon_1$ and $\epsilon_2$ from $\phi_1$ and $\phi_2$, respectively; we can satisfy the requisite condition using their infimum, $\min(\epsilon_1,\epsilon_2)>0$.

It remains to check condition 3. Let $I:(\tRR_j\to\Prop)\to\Prop$ satisfy $\forall(\phi:\tRR_j\to\Prop)\ldotp I(\phi)\imp\Op(\phi)$, i.e.\ $I$ is an open system. We need to show $\Op(\lambda(r:\tRR_j)\ldotp\exists(\phi:\tRR_j\to\Prop)\ldotp I(\phi)\wedge\phi (r))$. In other words, we need to show
\begin{multline*}
  \forall(r:\tRR_j)\ldotp[\exists(\phi:\tRR_j\to\Prop)\ldotp I(\phi)\wedge\phi(r)]\imp\exists(\epsilon:\tRR_j)\ldotp\epsilon>0\wedge\\
  \forall(r':\tRR_j)\ldotp(-\epsilon<r-r'<\epsilon)\imp\exists(\phi':\tRR_j\to\Prop)\ldotp I(\phi')\wedge\phi'(r).
\end{multline*}
So take $r:\tRR_j$ and suppose given $\phi$ such that $I(\phi)$ and $\phi(r)$ hold. It follow by hypothesis that $\Op(\phi)$ holds, so by definition we obtain $\epsilon>0$ satisfying the requisite condition with $\phi'\coloneqq\phi$.

All we used about $\tRR_j$ was that it has a definition of comparison ($a<b$), subtraction ($a-b$), that binary min's exist, and that the additive inverse of a binary min is a binary max.
\end{example}

\begin{remark}
\label{constants_open}
Every constant predicate $\lambda x.P$ for $P : \Prop$ is open: take $I(\phi) := \phi \land P$ in \Cref{internal_space}. 
\end{remark}

\begin{definition}\label{def.continuous}
Suppose that $(X,\Op_X)$ and $(Y,\Op_Y)$ are internal topological space. A function $f:X\to Y$ is called \emph{continuous} if it satisfies
\[\forall(\phi:Y\to\Prop)\ldotp\Op_Y(\phi)\imp\Op_X(\phi f).\]
We denote by $\Cont(X,Y)$ the type of all $f:X\to Y$ satisfying this condition.
\end{definition}

\begin{proposition}
The internal topological spaces in a topos $\cat{E}$ and continuous maps form an external category with products, and the forgetful functor to $\cat{E}$ preserves products.
\end{proposition}
\begin{proof}
It is easy to check directly from \cref{def.continuous} that if $f:X\to Y$ and $g:Y\to Z$ are continuous then so is $g\circ f$, and the identity is clearly continuous. To check the existence of products, we need to define a product topology $\Op_{X\times Y}$ on $X\times Y$, given topologies $\Op_X$ and $\Op_Y$. Given $\phi:X\times Y\to\Prop$, define $\Op_{X\times Y}(\psi)$ to be
\begin{multline*}
	\exists(I:\mathrm{OpSys}(X))\ldotp\\
	[\forall(\phi:I)\ldotp\exists(\phi_X:\Op(X))(\phi_Y:\Op(Y))\ldotp\phi(x,y)\iff(\phi_X(x)\wedge\phi_Y(y))]\\\wedge[\forall(x:X)(y:Y)\ldotp\psi(x,y)\iff\exists(\phi:I)\ldotp\phi(x,y)]
\end{multline*}
Check that this is a topology and that it has the universal property of a product. (Unfinished)
\end{proof}

\begin{example}
The functions $+:\tRR_j\times\tRR_j\to\tRR_j$ and $*:\tRR_j\times\tRR_j\to\tRR_j$ are continuous; see \cite[Theorems 4.44, 4.48]{Schultz.Spivak:2017a}. For any $j\to j'$ the map $\tRR_j\to\tRR_{j'}$ is also continuous; see \cite[Proposition 4.52]{Schultz.Spivak:2017a}.
\end{example}

%%% The following is the Riesz approach
% \begin{definition}[Measure on an internal topological space]
%Let $(X,\Op_X)$ be an internal topological space, and let $\tRR_{\SeeInline{[d,u]}}$ be the $[d,u]$-reals with the usual topology (see \cref{ex.usual_topology_R}). Then a $[d,u]$-measure on $X$ is a function $\mu:\Cont(X,\tRRat{[d,u]})\to\tRRat{[d,u]}$
%satisfying the following properties:
%\begin{enumerate}
%	\item Linearity: $\mu(r*f+g)=r*\mu(f)+\mu(g)$ for $r:\tRR$ and $f,g:X\to\tRRat{[d,u]}$.
%	\item Positivity: $\mu(f)\geq 0$ whenever $f$ satisfies $\forall x\ldotp f(x)\geq 0$.
%\end{enumerate}
%Let $\const{Meas}_{[d,u]}(X)$ denote the type of measures on $X$ at $[d,u]$. A \emph{measure $\mu$ on $X$} consists of a measure $\mu_{[d,u]}\in\const{Meas}_{[d,u]}(X)$ for each $d,u:\tRR$ such that
%\[\forall(f:X\to\tRRat{[d',u']})\ldotp\See{[d,u]}\mu_{[d',u']}(f)=\mu_{[d,u]}(\See{[d,u]}f)\]
%holds for every $d\leq d'\leq u'\leq u$. Let $\const{Meas}(X)$ denote the type of measures on $X$.
%\end{definition}

\tob{We may eventually need a discussion of locally compact spaces as continuous posets. Does at modality take locally compact spaces to locally compact spaces?}

\begin{proposition}
	\label{prop.internal_space_implication}
	If $(X,\Op(X))$ is an internal space, then $\Op(X)$ inherits implication (the Heyting exponential) from $\Prop^X$.
\end{proposition}

\begin{proof}
	For $P,Q : \Op(X)$, the Heyting implication in the frame $\Op(X)$ is given by
	\[
		(P \imp_{\Op(X)} Q) \: = \: \exists R : \Op(X) \ldotp (R \land P \imp Q) \land R,
	\]
	where it is easy to see that this has the desired adjointness in the form that for all $R : \Op(X)$,
	\[
		\left[ R \land P \imp Q \right] \quad \Longleftrightarrow \quad \left[ R \imp (P \imp_{\Op(X)} Q) \right],
	\]
	and we also use the fact that the above $P \imp_{\Op(X)} Q$ is also open thanks to the assumption that arbitrary unions of opens are open.

	Now we have
	\begin{align*}
		(P \imp_{\Op(X)} Q) \quad & \Leftrightarrow \quad \exists R : \Op(X) \ldotp (R \land P \imp Q) \land R \\
			& \Leftrightarrow \quad \exists R : \Op(X) \ldotp (P \imp Q) \land R \\
			& \Leftrightarrow \quad (P \imp Q),
	\end{align*}
	as was to be shown.
\end{proof}

\begin{lemma}
	\label{lem.internal_space_at}
	If $P : X \to \Prop$ is an open predicate in an internal space $(X,\Op(X))$, then so is $\At{K} P : X \to \Prop$.
\end{lemma}

\begin{proof}
	Use \Cref{prop.internal_space_implication} together with the formula $\At{K} P \, = \, (P \imp \At{K} \bot) \imp \At{K} \bot$ from \Cref{lem.at_double_neg}, and using that the constant predicate $\At{K} \bot$ is also open by \Cref{constants_open}.	
\end{proof}

Finally, taking the interior works as follows:

\begin{definition}\label{def.interior}
	If $(X,\Op(X))$ is an internal space and $P : X \to \Prop$ is a predicate, define its \emph{interior} $\interior(P)\colon X\to\Prop$ as follows:
	\[
		\interior(P) := \lambda x \ldotp \exists U : \Op(X) \ldotp U(x) \land (U \imp P).
	\]
\end{definition}

This is the largest open predicate below $P$ by definition.


\chapter{Flabbiness}

\tob{We need to explain \href{https://ncatlab.org/nlab/show/flabby+sheaf\#characterization_using_the_internal_language}{https://ncatlab.org/nlab/show/flabby+sheaf\#characterization\_using\_the\_internal\_language}}

We discuss flabbiness from the internal perspective of any topos $\mathcal{E}$.

\begin{definition}\label{def:flabby}
	Let $X$ be a type. Say that $P\colon X\to\Prop$ is \emph{subsingleton} if
	\begin{equation}\label{eqn.subsingleton}
		\subsing{P} \: := \: \forall (x,x' : X) \ldotp P(x) \wedge P(x') \imp x = x'
	\end{equation}
	holds.
\end{definition}

\begin{definition}
$X$ is \emph{flabby} if for every subsingleton $S\subseteq X$ there is $x\in X$ such that if $S$ is inhabited, then $x\in S$. In other words, $X$ is flabby if for every $P : X \to \Prop$,
	\begin{equation}\label{eq:flabby}
		\subsing{P} \: \Longrightarrow \: \exists x : X \ldotp (\exists y : X \ldotp P(y)) \imp P(x) .
	\end{equation}
\end{definition}

\begin{lemma}
$\Prop$ is flabby.
\end{lemma}

\begin{proof}
Suppose that we have $P : \Prop \to \Prop$ that satisfies the subsingleton condition \cref{eqn.subsingleton}. Let $U := \forall V \ldotp P(V) \imp V$. We need to show that if we have $W : \Prop$ with $P(W)$, then also $P(U)$.

So it is enough to prove $U = W$. Assuming $U$, we get $P(W) \imp W$ as an instance of $U$, and therefore $W$. Assuming $W$, we can derive $U$ by proving $P(V) \imp V$ for a given $V$. But since $P(V)$ and $P(W)$, we get $V = W$ by the assumed subsingleton condition, and therefore $V$, as was to be shown.
\end{proof}

More generally, we have~\cite[Lemma~1.2]{moerdijk}. 
Recall from \cref{def.interior} the notion of the interior of a predicate.


\begin{proposition}
	\label{prop:opens_flabby}
	For every internal space $(X,\Op(X))$, the type $\Op(X)$ is flabby.
\end{proposition}

\begin{proof}
This is analogous to the previous proof. So suppose that we have $P : \Op(X) \to \Prop$ that satisfies the subsingleton condition \cref{eqn.subsingleton}. Then we claim that $P$ holds for $U := \interior\left(\forall V \ldotp P(V) \imp V\right)$. Hence we need to show that if we have $W : \Op(X)$ with $P(W)$, then also $P(U)$.

	So it is enough to prove $U = W$ by showing $U(x) \Leftrightarrow W(x)$ for every $x : X$. Assuming $U(x)$, we have an open $U'\ni x$ such that $U' \imp (\forall V \ldotp P(V) \imp V)$. Evaluating this at $x$ proves $\forall V. P(V) \imp V(x)$, and therefore we indeed get $W(x)$. Now assume $W(x)$. Using $W$ itself as the neighborhood, it is enough to prove that $W$ implies $\forall V\ldotp P(V) \imp V$. But now since $P(V)$ and $P(W)$, this follows from the assumed subsingleton condition.
\end{proof}

\begin{definition}\label{def.at_flabby}
	Let $K : \CB$. An inhabited type $X$ is \emph{$\At{K}$-flabby} if for all $P : X \to \Prop$,
	%\[
	%	\At{K}\bot \: \lor \: \left[ \subsing{P} \land \At{K} \exists x : X \ldotp P(x) \quad \Longrightarrow \quad \exists x : X \ldotp (\exists y : X \ldotp P(y)) \imp \At{K} P(x)  \right].
	%\]
	\begin{equation}
		\label{eq.at_flabby}
		\left( \At{K} \exists x : X \ldotp P(x)\right) \: \Longrightarrow \: \subsing{P} \: \Longrightarrow \: \exists x : X \ldotp \left[ (\exists y : X \ldotp P(y)) \imp \At{K} P(x) \right] .
	\end{equation}
\end{definition}

Clearly if $X$ is flabby, then it is also $\At{K}$-flabby for every $K$.

\begin{proposition}
	\label{prop.at_flabby_semantics}
	In our intended semantics of sheaves on $\CB$, an inhabited sheaf $X$ is $\At{K}$-flabby if only if every germ at $K$ can be extended to a global section.
\end{proposition}

\tob{Carefully check the proof again}

\begin{proof}
	The semantics of this internal condition on some open $U$ is as follows. The first assumption is that there is a smaller neighborhood $V\ni K$ and $x_0 \in X(V)$ such that $P(x_0) = V$; in other words, that $P$ contains a germ at $K$. The second assumption $U \models \subsing{P}$ means that for every $U'\subseteq U$, if $x,x'\in X(U')$ and $P(x) = P(x') = U'$, then $x = x'$. The conclusion of the condition then is $U \models \exists x : X \ldotp (\exists y : X \ldotp P(y)) \imp \At{K} P(x)$, meaning that there is a cover $U = \bigcup_i U_i$ and local sections $x_i \in X(U_i)$ such that for every $V_i \subseteq U_i$ which contains $K$, if $P$ contains a germ of $X$ at every point of $V_i$, then also $K\in P(x)$.

	We prove this implication from inhabitedness together with the possibility of lifting germs at $K$ to global sections. If $K\not\in U$, then the condition trivially holds upon choosing any $x$ (i.e.~covering family of local sections). So assume $K\in U$. The first assumption gives us a germ at $K$, which we extend to a section $\wh{x}\in X(U)$. For the given cover $U = \bigcup_i U_i$, we put $x_i := \wh{x}|_{U_i}$. Now we are given the $V_i \subseteq U_i$, and need to assume that $P$ contains a germ at every point of $V_i$. For $K\in V_i$, the subsingleton assumption implies that all of these germs are equal to the restriction of $x_i$. In particular, the germ at $K$, which must be equal to the restriction of $x$, satisfies $P$ in a neighborhood of $K$; hence $K\in P(x)$, as was to be shown.

	Conversely, suppose that the internal sentence holds. Now it is enough to show that every germ at $K$ can be lifted to a germ at a given $K'\supseteq K$; for then, Zorn's lemma shows that the given germ can be extended to an open which contains all these $K'$; but then the only such open is the whole space $\CB$, resulting in a global section \todo{this makes some assumption on $\CB$}. So hence let $x \in X(U)$ be a representative of a germ defined on an open $U$ with $K \in U$. We instantiate the internal assumption on the open $\CB$. We define a subsingleton as the natural transformation $P : X \to \Prop$ with components $P_V(x') := \bigcup \{V' \subseteq V \: : \: x'|_{V'} = x|_{V'} \}$, which is clearly natural; also the subsingleton condition holds by construction. Also the other assumption $\CB \models \At{K} \exists x : X \ldotp P(x)$ holds by construction. Thus at every $K' \supseteq K$, we obtain a germ $x'$ defined on a neighborhood $U' \ni K'$ which necessarily also contains $K$. Since $U'\models (\exists y : X \ldotp P(y) ) \imp \At{K} P(x')$, we conclude $P(x') \ni K$ by restriction to the neighborhood of $K$ on which $x$ is defined. Restricting to the intersection of the two domains of definition of $x$ and $x'$, we conclude that $x = x'$ holds there. Hence $x'$ extends the germ represented by $x$, as was to be shown.
\end{proof}

\begin{lemma}\label{lemma.11_flabby}
	$1 + 1$ is $\At{K}$-flabby.
	\label{lem.11_at_flabby}
\end{lemma}

\begin{proof}
	$1 + 1$ is trivially inhabited, so we only prove the main condition~\eqref{eq.at_flabby}.

	Let us write $1 + 1 = \{\alpha,\beta\}$ in terms of elements. Let $P : 1 + 1 \to \Prop$, and assume $\At{K}\exists(x:1+1)\ldotp P(x)$. From the fact that $\forall(x:1+1)\ldotp (x=\alpha)\vee (x=\beta)$, we have $\At{K}P(\alpha)\vee \At{K}P(\beta)$.
	In case $\At{K} P(\alpha)$, we use $x := \alpha$ and there is nothing more to prove; similarly $x := \beta$ if $\At{K} P(\beta)$. %Hence we can assume $P(\alpha) \imp \At{K} \bot$ and $P(\beta) \imp \At{K} \bot$. But now we also have the additional assumption $\At{K}(P(\alpha) \lor P(\beta))$, which now implies $\At{K} \bot$, which is enough.
\end{proof}
Note that we did not use the subsingleton condition on $P$ in the proof of \cref{lemma.11_flabby}.

%\printbibliography

\end{document}
